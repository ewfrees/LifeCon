\documentclass[]{book}
\usepackage{lmodern}
\usepackage{amssymb,amsmath}
\usepackage{ifxetex,ifluatex}
\usepackage{fixltx2e} % provides \textsubscript
\ifnum 0\ifxetex 1\fi\ifluatex 1\fi=0 % if pdftex
  \usepackage[T1]{fontenc}
  \usepackage[utf8]{inputenc}
\else % if luatex or xelatex
  \ifxetex
    \usepackage{mathspec}
  \else
    \usepackage{fontspec}
  \fi
  \defaultfontfeatures{Ligatures=TeX,Scale=MatchLowercase}
\fi
% use upquote if available, for straight quotes in verbatim environments
\IfFileExists{upquote.sty}{\usepackage{upquote}}{}
% use microtype if available
\IfFileExists{microtype.sty}{%
\usepackage{microtype}
\UseMicrotypeSet[protrusion]{basicmath} % disable protrusion for tt fonts
}{}
\usepackage[margin=1in]{geometry}
\usepackage{hyperref}
\hypersetup{unicode=true,
            pdftitle={Actuarial Mathematics Analytics},
            pdfauthor={An open text authored by the Actuarial Community},
            pdfborder={0 0 0},
            breaklinks=true}
\urlstyle{same}  % don't use monospace font for urls
\usepackage{natbib}
\bibliographystyle{apalike}
\usepackage{longtable,booktabs}
\usepackage{graphicx,grffile}
\makeatletter
\def\maxwidth{\ifdim\Gin@nat@width>\linewidth\linewidth\else\Gin@nat@width\fi}
\def\maxheight{\ifdim\Gin@nat@height>\textheight\textheight\else\Gin@nat@height\fi}
\makeatother
% Scale images if necessary, so that they will not overflow the page
% margins by default, and it is still possible to overwrite the defaults
% using explicit options in \includegraphics[width, height, ...]{}
\setkeys{Gin}{width=\maxwidth,height=\maxheight,keepaspectratio}
\IfFileExists{parskip.sty}{%
\usepackage{parskip}
}{% else
\setlength{\parindent}{0pt}
\setlength{\parskip}{6pt plus 2pt minus 1pt}
}
\setlength{\emergencystretch}{3em}  % prevent overfull lines
\providecommand{\tightlist}{%
  \setlength{\itemsep}{0pt}\setlength{\parskip}{0pt}}
\setcounter{secnumdepth}{5}
% Redefines (sub)paragraphs to behave more like sections
\ifx\paragraph\undefined\else
\let\oldparagraph\paragraph
\renewcommand{\paragraph}[1]{\oldparagraph{#1}\mbox{}}
\fi
\ifx\subparagraph\undefined\else
\let\oldsubparagraph\subparagraph
\renewcommand{\subparagraph}[1]{\oldsubparagraph{#1}\mbox{}}
\fi

%%% Use protect on footnotes to avoid problems with footnotes in titles
\let\rmarkdownfootnote\footnote%
\def\footnote{\protect\rmarkdownfootnote}

%%% Change title format to be more compact
\usepackage{titling}

% Create subtitle command for use in maketitle
\newcommand{\subtitle}[1]{
  \posttitle{
    \begin{center}\large#1\end{center}
    }
}

\setlength{\droptitle}{-2em}
  \title{Actuarial Mathematics Analytics}
  \pretitle{\vspace{\droptitle}\centering\huge}
  \posttitle{\par}
  \author{An open text authored by the Actuarial Community}
  \preauthor{\centering\large\emph}
  \postauthor{\par}
  \predate{\centering\large\emph}
  \postdate{\par}
  \date{2017-06-16}

\usepackage{amssymb,amsmath}
\usepackage{lifecon}
%\usepackage{mathspec}
%\makeatletter % undo the wrong changes made by mathspec
%\let\RequirePackage\original@RequirePackage
%\let\usepackage\RequirePackage
%\makeatother

\usepackage{booktabs}
\setcounter{secnumdepth}{2}

\begin{document}
\maketitle

{
\setcounter{tocdepth}{2}
\tableofcontents
}
\chapter*{Preface}\label{preface}
\addcontentsline{toc}{chapter}{Preface}

\subsubsection*{Book Description}\label{book-description}
\addcontentsline{toc}{subsubsection}{Book Description}

\textbf{Actuarial Mathematics Analytics} is an interactive, online,
freely available text.

\begin{itemize}
\item
  The online version contains many interactive objects (quizzes,
  computer demonstrations, interactive graphs, video, and the like) to
  promote \emph{deeper learning}.
\item
  A subset of the book is available for \emph{offline reading} in pdf
  and EPUB formats.
\item
  The online text will be available in multiple languages to promote
  access to a \emph{worldwide audience}.
\end{itemize}

\subsubsection*{What will success look
like?}\label{what-will-success-look-like}
\addcontentsline{toc}{subsubsection}{What will success look like?}

The online text will be freely available to a worldwide audience. The
online version will contain many interactive objects (quizzes, computer
demonstrations, interactive graphs, video, and the like) to promote
deeper learning. Moreover, a subset of the book will be available in pdf
format for low-cost printing. The online text will be available in
multiple languages to promote access to a worldwide audience.

\subsubsection*{How will the text be
used?}\label{how-will-the-text-be-used}
\addcontentsline{toc}{subsubsection}{How will the text be used?}

This book will be useful in actuarial curricula worldwide. It will cover
the loss data learning objectives of the major actuarial organizations.
Thus, it will be suitable for classroom use at universities as well as
for use by independent learners seeking to pass professional actuarial
examinations. Moreover, the text will also be useful for the continuing
professional development of actuaries and other professionals in
insurance and related financial risk management industries.

\subsubsection*{Why is this good for the
profession?}\label{why-is-this-good-for-the-profession}
\addcontentsline{toc}{subsubsection}{Why is this good for the
profession?}

An online text is a type of open educational resource (OER). One
important benefit of an OER is that it equalizes access to knowledge,
thus permitting a broader community to learn about the actuarial
profession. Moreover, it has the capacity to engage viewers through
active learning that deepens the learning process, producing analysts
more capable of solid actuarial work. Why is this good for students and
teachers and others involved in the learning process?

Cost is often cited as an important factor for students and teachers in
textbook selection (see a recent post on the
\href{https://www.aei.org/publication/the-new-era-of-the-400-college-textbook-which-is-part-of-the-unsustainable-higher-education-bubble/}{\$400
textbook}). Students will also appreciate the ability to ``carry the
book around'' on their mobile devices.

\subsubsection*{Why loss data analytics?}\label{why-loss-data-analytics}
\addcontentsline{toc}{subsubsection}{Why loss data analytics?}

Although the intent is that this type of resource will eventually
permeate throughout the actuarial curriculum, one has to start
somewhere. Given the dramatic changes in the way that actuaries treat
data, loss data seems like a natural place to start. The idea behind the
name \emph{loss data analytics} is to integrate classical loss data
models from applied probability with modern analytic tools. In
particular, we seek to recognize that big data (including social media
and usage based insurance) are here and high speed computation s readily
available.

\subsubsection*{Project Goal}\label{project-goal}
\addcontentsline{toc}{subsubsection}{Project Goal}

The project goal is to have the actuarial community author our textbooks
in a collaborative fashion.

To get involved, please visit our
\href{https://sites.google.com/a/wisc.edu/loss-data-analytics/}{Loss
Data Analytics Project Site}.

\chapter{Survival Models and Life
Tables}\label{survival-models-and-life-tables}

Let (x) denote a life aged x.

\section{Future Lifetime}\label{future-lifetime}

\begin{itemize}
\item
  \(T_x\) = \textbf{time-until-death for (x)}, a continuous random
  variable (in years).

  \(T_x\) is also called the \textbf{future lifetime random variable}.
  \(T_x\) may also be written as \(T(x)\) or \(T\).

  Special case: \(T_0\) = \textbf{age-at-death for (0)}, where (0)
  denotes a newborn life. Note: \(T_0\) = x + \(T_x\), given survival to
  age x.
\item
  \(F_x(t)\) = \({}_{t}q_x\) = \(Pr(T_x \le t)\)

  This is the \textbf{cumulative distribution function of \(T_x\)},
  ``the probability that (x) dies within \(t\) years.'' The
  \(q\)-notation will be used most of the time.

  \(F_0(t)\) can also be written more simply as \(F(t)\).
\item
  \(S_x(t)\) = \({}_{t}p_x\) = \(Pr(T_x > t)\)

  This is the \textbf{survival function function of \(T_x\)},``the
  probability that (x) survives for at least \(t\) years.'' The words
  ``at least'' are often omitted. The \(p\)-notation will be used most
  of the time.

  \(S_0(t)\) can also be written more simply as \(S(t)\) or \(s(t)\).
\item
  From above: \({}_{t}q_x\) + \({}_{t}p_x\) = 1.

  ``(x) will either survive or die within \(t\) years.''
\item
  \(S_0(x + t)\) = \(S_0(x)\)\({}_{t}p_x\)

  ``The probability that (0) survives \(x + t\) years is equivalent to
  (0) first surviving \(x\) years to age x, and then surviving \(t\)
  additional years to age x + t.''
\end{itemize}

\begin{itemize}
\item
  \({}_{u + t}p_x\) = \(({}_{u}p_x)\)\(({}_{t}p_{x + u})\)

  ``The probability that (x) survives \(u + t\) years is equivalent to
  (x) first surviving \(u\) years to age x + u, and then surviving \(t\)
  additional years to age x + u + t.''
\item
  Be careful: \({}_{u + t}q_x\) \(\neq\)
  \(({}_{u}q_x)\)\(({}_{t}q_{x + u})\).

  ``The right-hand side implies that it is possible for (x) to die
  within \(u\) years, then somehow come back to life at age x + u in
  order to die again within \(t\) years. This, of course, is not
  possible and cannot be equal to the left-hand side which is the
  probability (x) dies within \(u + t\) years.''
\item
  \({}_{u|t}q_x\) = \({}_{u + t}q_x\) - \({}_{u}q_x\) = \({}_{u}p_x\) -
  \({}_{u + t}p_x\) = (\({}_{u}p_x\))(\({}_{t}q_{x + u}\))

  This is a \(u\)-year deferred probability of death, ``the probability
  that (x) dies between ages x + u and x + u + t.'' Note:
  \({}_{0|t}q_x\) = \({}_{t}q_x\).
\item
  Note: \({}_{1}q_x\), \({}_{1}p_x\), and \({}_{u|1}q_x\) are written as
  \(q_x\), \(p_x\), and \({}_{u|}q_x\), respectively.

  \(q_x\) may be referred to as a \textbf{mortality rate} at age x, and
  \(p_x\) may be referred to as a \textbf{survival rate} at age x.
\end{itemize}

\section{Force of Mortality}\label{force-of-mortality}

\begin{itemize}
\item
  \(\mu_x\) = \(\mu(x)\) = \textbf{force of mortality at age x}, given
  survival to age x. This is sometimes called the ``hazard rate'' or
  ``failure rate.''

  \(\mu_x\) = \(-\frac{\frac{d}{dx}[S_0(x)]}{S_0(x)}\) =
  \(-\frac{d}{dx}[\ln S_0(x)]\)

  ``This is the instantaneous death rate for a life at age x.''
\item
  \(\mu_{x + t}\) = \(\mu_x(t)\) = \textbf{force of mortality at age x +
  t}, given survival to x + t.

  \(\mu_{x + t}\) = \(-\frac{\frac{d}{dt}[{}_{t}p_x]}{{}_{t}p_x}\) =
  \(-\frac{d}{dt}[\ln {}_{t}p_x]\)

  ``This is the instantaneous death rate for a life at age x + t. Here,
  the variable is time after age x. You could also obtain
  \(\mu_{x + t}\) by replacing x in \(\mu_x\) with x + t.''
\item
  \({}_{t}p_x\) = \(\exp[-\int^{x + t}_x \mu_sds]\) =
  \(\exp[-\int^{t}_0 \mu_{x + s}ds]\)
\item
  If \(c\) \(>\) 0, then \(\mu^{*}_{x + s}\) = \(c\mu_{x + s}\)
  \(\implies\) \({}_{t}p^{*}_x\) = \(({}_{t}p_x)^{c}\).

  For constant \(k\), then \(\mu^{*}_{x + s}\) = \(\mu_{x + s}\) + \(k\)
  \(\implies\) \({}_{t}p^{*}_x\) = \((e^{-kt})({}_{t}p_x)\).

  The constant \(k\) should be such that \(\mu^{*}_{x + s}\) \(>\) 0.
\item
  \(f_x(t)\) = \({}_{t}p_x\)\(\mu_{x + t}\) = \textbf{probability
  density function of \(T_x\)}.

  This comes from the above formula for \(\mu_{x + t}\), recognizing
  that \(f_x(t)\) = \(\frac{d}{dt}[{}_{t}q_x]\) =
  \(-\frac{d}{dt}[{}_{t}p_x]\).
\item
  \({}_{t}q_x\) = \(\int^t_0 {}_{s}p_x\)\(\mu_{x + s} ds\)
\item
  \({}_{t}p_x\) = \(\int^{\infty}_t {}_{s}p_x\)\(\mu_{x + s} ds\)
\item
  \({}_{u|t}q_x\) = \(\int^{u+t}_u {}_{s}p_x\mu_{x + s}ds\)
\end{itemize}

\section{Curtate Future Lifetime}\label{curtate-future-lifetime}

\begin{itemize}
\item
  \(K_x\) = \textbf{curtate future lifetime for (x)}, a discrete random
  variable.

  \(K_x\) = \(\left\lfloor{T_x}\right\rfloor\) = integer part of
  \(T_x\). That is, \(K_x\) represents the complete number of future
  years survived by (x), where any fractional time survived in the year
  of death is ignored. Note: \(K_x\) = 0, 1, 2\ldots{}

  \(K_x\) may also be written as \(K(x)\) or \(K\).
\item
  \({}_{k|}q_{x}\) = \(Pr(K_x = k)\) = \(Pr(k \le T_x < k + 1)\) for
  \(k\) = 0, 1, 2\ldots{}

  This is the \textbf{probability mass function of \(K_x\)}, ``the
  probability that (x) dies in the (\(k\) + 1)st year, between ages x +
  k and x + k + 1.''
\item
  \({}_{k+1}q_{x}\) = \({}_{0|}q_{x}\) + \({}_{1|}q_{x}\) + \ldots{} +
  \({}_{k|}q_{x}\)

  ``The probability that (x) dies within \(k\) + 1 years is the sum of
  the probabilities that (x) dies in the first year, the second year,
  \ldots{}, the (\(k\) + 1)st year.''
\end{itemize}

\section{\texorpdfstring{Other Features of \(T_x\) and \(K_x\)
Distributions}{Other Features of T\_x and K\_x Distributions}}\label{other-features-of-t_x-and-k_x-distributions}

\begin{itemize}
\item
  \(\mathring{e}_x\) = \(E(T_x)\) =
  \(\int^{\infty}_0 t({}_{t}p_x)(\mu_{x + t})dt\) =
  \(\int^{\infty}_0 {}_{t}p_xdt\)

  This is the \textbf{complete expectation of life for (x)}, the average
  time-until-death for (x). That is, (x) is expected to die at age x +
  \(\mathring{e}_x\).
\item
  \(Var(T_x)\) = \(E(T_x^2)\) - \([E(T_x)]^2\) =
  \(2\int^{\infty}_0 t{}_{t}p_xdt\) - \([\mathring{e}_x]^2\)
\item
  \(\mathring{e}_{x :\overline{n}|}\) = \(E[min(T_x, n)]\) =
  \(\int^{n}_0 {}_{t}p_xdt\)

  This is the \textbf{n-year temporary complete life expectancy for
  (x)}, the average number of years out of the next n years that (x)
  survives.

  This expectation helps define the recursion: \(\mathring{e}_x\) =
  \(\mathring{e}_{x :\overline{n}|}\) +
  \({}_{n}p_x\)\(\mathring{e}_{x + n}\).

  ``The average number of future years that (x) survives is the average
  number of years out of the first n years that (x) survives plus the
  average number of years (x) survives after the first n years
  (accounting for the probability that (x) survives the first n
  years).''

  Similarly: \(\mathring{e}_{x :\overline{m + n}|}\) =
  \(\mathring{e}_{x :\overline{m}|}\) +
  \({}_{m}p_x\)\(\mathring{e}_{x + m :\overline{n}|}\).
\item
  The \textbf{100\(\alpha\)-th percentile of the distribution of
  \(T_x\)}, \(\pi_{\alpha}\), is such that:

  \({}_{\pi_{\alpha}}q_x\) = \(\alpha\) for 0 \(\le\) \(\alpha\) \(\le\)
  1.

  Special case: \(\alpha\) = 0.50; \(\pi_{.50}\) is called the
  \textbf{median future lifetime for (x)}.
\end{itemize}

\begin{itemize}
\item
  \(e_x\) = \(E(K_x)\) = \(\sum^{\infty}_{k = 0} k({}_{k|}q_x)\) =
  \(\sum^{\infty}_{k = 1} {}_{k}p_x\)

  This is the \textbf{curtate expectation of life for (x)}, the average
  curtate future lifetime for (x).
\item
  \(\mathring{e}_x\) \(\approx\) \(e_x\) + \(\frac{1}{2}\)
\item
  \(Var(K_x)\) = \(E(K_x^2)\) - \([E(K_x)]^2\) =
  \(\sum^{\infty}_{k = 1} (2k - 1){}_{k}p_x\) - \([e_x]^2\)
\item
  \(e_{x :\overline{n}|}\) = \(E[min(K_x, n)]\) =
  \(\sum^{n}_{k = 1} {}_{k}p_x\)

  This is the \textbf{n-year temporary curtate life expectancy for (x)}.

  This expectation helps define the recursions: \(e_x\) =
  \(e_{x :\overline{n}|}\) + \({}_{n}p_x\)\(e_{x + n}\) and

  \(e_{x :\overline{m + n}|}\) = \(e_{x :\overline{m}|}\) +
  \({}_{m}p_x\)\(e_{x + m :\overline{n}|}\).
\end{itemize}

\section{Special Mortality Laws}\label{special-mortality-laws}

\subsection{de Moivre's Law}\label{de-moivres-law}

\(T_x\) has a continuous uniform distribution.

The limiting age is \(\omega\) such that 0 \(\le\) \(x\) \(\le\)
\(x + t\) \(\le\) \(\omega\).

\begin{itemize}
\item
  \(\mu_x\) = \(\frac{1}{\omega - x}\) (Note: \(x\) \(\neq\) \(\omega\))
\item
  \(S_0(x)\) = \(\frac{\omega - x}{\omega}\)
\item
  \(F_0(x)\) = \(\frac{x}{\omega}\)
\item
  \({}_{t}p_x\) = \(\frac{\omega - x - t}{\omega - x}\)
\item
  \({}_{t}q_x\) = \(\frac{t}{\omega - x}\)
\item
  \({}_{u|t}q_x\) = \(\frac{t}{\omega - x}\)
\item
  \(f_x(t)\) = \({}_{t}p_x\)\(\mu_{x + t}\) = \(\frac{1}{\omega - x}\)
  (Note: \(x\) \(\neq\) \(\omega\))
\item
  \(\mathring{e}_x\) = \(\frac{\omega - x}{2}\)
\item
  \(\mathring{e}_{x :\overline{n}|}\) = \(n{}_{n}p_x\) +
  \(\frac{n}{2}{}_{n}q_x\)

  ``(x) can either survive \(n\) years with probability \({}_{n}p_x\),
  or die within \(n\) years with probability \({}_{n}q_x\). Surviving
  \(n\) years contributes \(n\) to the expectation. Dying within \(n\)
  years contributes \(\frac{n}{2}\) to the expectation as future
  lifetime has a uniform distribution - (x), on average, would die
  halfway through the \(n\)-year period.''
\item
  \(Var(T_x)\) = \(\frac{(\omega - x)^2}{12}\)
\item
  \(e_x\) = \(\frac{\omega - x - 1}{2}\)
\item
  \(Var(K_x)\) = \(\frac{(\omega - x)^2}{12}\) - \(\frac{1}{12}\)
\end{itemize}

\subsection{Modified/Generalized de Moivre's
Law}\label{modifiedgeneralized-de-moivres-law}

\(T_x\) has a beta distribution.

The limiting age is \(\omega\) such that 0 \(\le\) \(x\) \(\le\)
\(x + t\) \(\le\) \(\omega\). Also, there is a parameter \(\alpha\)
\(>\) 0.

\begin{itemize}
\item
  \(\mu_x\) = \(\frac{\alpha}{\omega - x}\) (Note: \(x\) \(\neq\)
  \(\omega\))
\item
  \(S_0(x)\) = \(\left(\frac{\omega - x}{\omega}\right)^{\alpha}\)
\item
  \(F_0(x)\) = 1 - \(\left(\frac{\omega - x}{\omega}\right)^{\alpha}\)
\item
  \({}_{t}p_x\) =
  \(\left(\frac{\omega - x - t}{\omega - x}\right)^{\alpha}\)
\item
  \({}_{t}q_x\) = 1 -
  \(\left(\frac{\omega - x - t}{\omega - x}\right)^{\alpha}\)

  Note: \({}_{t}q_x\) \(\neq\)
  \(\left(\frac{t}{\omega - x}\right)^{\alpha}\).
\item
  \(\mathring{e}_x\) = \(\frac{\omega - x}{\alpha + 1}\)
\item
  \(Var(T_x)\) =
  \(\frac{\alpha(\omega - x)^2}{(\alpha + 1)^2(\alpha + 2)}\)

  Note: \(\alpha\) = 1 results in uniform distribution/de Moivre's Law.
\end{itemize}

\subsection{Constant Force of
Mortality}\label{constant-force-of-mortality}

\(T_x\) has an exponential distribution, \(x\) \(\ge\) 0. There is
another parameter that denotes the force of mortality: \(\mu\) \(>\) 0.

\begin{itemize}
\item
  \(\mu_x\) = \(\mu\)
\item
  \(S_0(x)\) = \(e^{-\mu x}\)
\item
  \(F_0(x)\) = 1 - \(e^{-\mu x}\)
\item
  \({}_{t}p_x\) = \(e^{-\mu t}\) = \((p_x)^t\)
\item
  \({}_{t}q_x\) = 1 - \(e^{-\mu t}\)
\item
  \(\mathring{e}_x\) = \(\frac{1}{\mu}\)
\item
  \(\mathring{e}_{x :\overline{n}|}\) = \(\frac{1 - e^{-\mu n}}{\mu}\)
\item
  \(Var(T_x)\) = \(\frac{1}{\mu^2}\)
\item
  \(e_x\) = \(\frac{p_x}{q_x}\)
\item
  \(Var(K_x)\) = \(\frac{p_x}{(q_x)^2}\)

  Note1: A constant force of mortality implies that ``age does not
  matter.'' This can easily be seen from \({}_{t}p_x\) = \(e^{-\mu t}\);
  x does not appear on the right-hand side.

  Note2: \(T_x\) has an exponential distribution implies that \(K_x\)
  has a geometric distribution.
\end{itemize}

\subsection{Gompertz's Law}\label{gompertzs-law}

\begin{itemize}
\item
  \(\mu_x\) = \(Bc^x\) for \(x\) \(\ge\) 0, \(B\) \(>\) 0, \(c\) \(>\) 1
\item
  \(S_0(x)\) = \(\exp[-\frac{B}{\ln c}(c^x - 1)]\)
\item
  \(F_0(x)\) = 1 - \(\exp[-\frac{B}{\ln c}(c^x - 1)]\)
\item
  \({}_{t}p_x\) = \(\exp[-\frac{B}{\ln c}c^x(c^t - 1)]\)
\item
  \({}_{t}q_x\) = 1 - \(\exp[-\frac{B}{\ln c}c^x(c^t - 1)]\)
\end{itemize}

Note: \(c\) = 1 results in a constant force of mortality.

\subsection{Makeham's Law}\label{makehams-law}

\begin{itemize}
\item
  \(\mu_x\) = \(A\) + \(Bc^x\) for \(x\) \(\ge\) 0, \(A\) \(\ge\)
  -\(B\), \(B\) \(>\) 0, \(c\) \(>\) 1
\item
  \(S_0(x)\) = \(\exp[-Ax - \frac{B}{\ln c}(c^x - 1)]\)
\item
  \(F_0(x)\) = 1 - \(\exp[-Ax - \frac{B}{\ln c}(c^x - 1)]\)
\item
  \({}_{t}p_x\) = \(\exp[-At - \frac{B}{\ln c}c^x(c^t - 1)]\)
\item
  \({}_{t}q_x\) = 1 - \(\exp[-At - \frac{B}{\ln c}c^x(c^t - 1)]\)
\end{itemize}

Note1: \(A\) = 0 results in Gompertz's Law.

Note2: \(c\) = 1 results in a constant force of mortality.

\subsection{Weibull's Law}\label{weibulls-law}

\(T_x\) has a Weibull distribution.

\begin{itemize}
\item
  \(\mu_x\) = \(k\)\(x^n\) for \(x\) \(\ge\) 0, \(k\) \(>\) 0, \(n\)
  \(>\) 0
\item
  \(S_0(x)\) = \(\exp[-\frac{k}{n + 1}x^{n + 1}]\)
\item
  \(F_0(x)\) = 1 - \(\exp[-\frac{k}{n + 1}x^{n + 1}]\)
\item
  \({}_{t}p_x\) =
  \(\exp[-\frac{k}{n + 1}((x + t)^{n + 1} - x^{n + 1})]\)
\item
  \({}_{t}q_x\) = 1 -
  \(\exp[-\frac{k}{n + 1}((x + t)^{n + 1} - x^{n + 1})]\)
\end{itemize}

\section{Life Tables}\label{life-tables}

Given a survival model with survival probabilities \({}_{t}p_x\), one
can construct a \textbf{life table}, also called a \textbf{mortality
table}, from some initial age \(x_0\) (usually age 0) to a maximum age
\(\omega\) (a limiting age).

\begin{itemize}
\item
  Let \(l_{x_0}\), the \textbf{radix} of the life table, represent the
  number of lives age \(x_0\).

  \(l_{x_0}\) is an arbitrary positive number.
\item
  \(l_{\omega}\) = 0.
\item
  \(l_{x + t}\) = \((l_x)({}_{t}p_x)\) for \(x_0\) \(\le\) \(x\) \(\le\)
  \(x + t\) \(\le\) \(\omega\).

  \(l_{x + t}\) represents the \textbf{expected number of survivors} to
  age x + t out of \(l_x\) individuals aged x.
\item
  \({}_{t}d_x\) = \(l_x\) - \(l_{x + t}\) = \((l_x)({}_{t}q_x)\) for
  \(x_0\) \(\le\) \(x\) \(\le\) \(x + t\) \(\le\) \(\omega\).

  \({}_{t}d_x\) represents the \textbf{expected number of deaths}
  between ages x and x + t out of \(l_x\) lives aged x.

  Note 1: \({}_{1}d_x\) is written as \(d_x\).

  Note 2: If \(n\) = 1, 2\ldots{}, then \({}_{n}d_x\) = \(d_x\) +
  \(d_{x + 1}\) + \ldots{} + \(d_{x + n - 1}\).
\item
  \({}_{t}d_{x + u}\) = \(l_{x + u}\) - \(l_{x + u + t}\) =
  \((l_x)({}_{u|t}q_x)\).
\end{itemize}

The \textbf{Illustrative Life Table} is the life table that is provided
to the candidate taking Exam MLC. Some questions from either exam will
involve Illustrative Life Table calculations. A web link to this table
(and ALL exam tables) is provided for each exam in Appendix A of this
study supplement.

\section{Fractional Age Assumptions}\label{fractional-age-assumptions}

Life Tables are usually defined for integer ages x and integer times t.
For a quantity that involves fractional ages and/or fractional times,
one has to make an assumption about the survival distribution between
integer ages; that is, one has to interpolate the value of the quantity
within each year of age. Two common interpolation assumptions follow.

\subsection{Uniform Distribution of Deaths
(UDD)}\label{uniform-distribution-of-deaths-udd}

One linearly interpolates within each year of age. For integer age x and
0 \(\le\) \(s\) \(\le\) \(s + t\) \(\le\) 1:

\begin{itemize}
\item
  \(l_{x + s}\) = \(l_x\) - \(s\)\(d_x\) = \((1 - s)l_x\) +
  \((s)l_{x + 1}\). This is a linear function of \(s\).
\item
  \({}_{s}q_x\) = \(s\)\(q_x\)
\item
  \({}_{s}p_x\) = 1 - \(s\)\(q_x\)
\item
  \(\mu_{x + s}\) = \(\frac{q_x}{1 - sq_x}\) (does not hold at \(s\) =
  1)
\item
  \(f_x(s)\) = \({}_{s}p_x\)\(\mu_{x + s}\) = \(q_x\) (does not hold at
  \(s\) = 1)
\item
  \({}_{s}q_{x + t}\) = \(\frac{sq_x}{1 - tq_x}\)
\item
  \(\mathring{e}_x\) = \(e_x\) + \(\frac{1}{2}\)
\item
  \(Var(T_x)\) = \(Var(K_x)\) + \(\frac{1}{12}\)
\item
  Note: uniform distribution/de Moivre's Law has the property of UDD
  across all ages up to the limiting age \(\omega\).

  Furthermore, uniform distribution/de Moivre's Law may be expressed as
  \(l_x\) = \(k(\omega - x)\) for 0 \(\le\) \(x\) \(\le\) \(\omega\)
  where \(k\) \(>\) 0.
\end{itemize}

\subsection{Constant Force of
Mortality}\label{constant-force-of-mortality-1}

One exponentially interpolates within each year of age. For integer age
x and 0 \(\le\) \(s\) \(\le\) \(s + t\) \(\le\) 1:

\begin{itemize}
\item
  \(l_{x + s}\) = \(l_x\)\(p^s_x\) \(\implies\) \(\ln[l_{x + s}]\) =
  \((1 - s)\ln[l_x]\) + \(s\ln[l_{x + 1}]\). This is an exponential
  function of \(s\).
\item
  \({}_{s}p_x\) = \(p^s_x\)
\item
  \({}_{s}q_x\) = 1 - \(p^s_x\)
\item
  \(\mu_{x + s}\) = -\(\ln p_x\) = \(\mu_x\) (does not hold at \(s\) =
  1)
\item
  \(f_x(s)\) = \({}_{s}p_x\)\(\mu_{x + s}\) = -\(\ln p_x\)\((p^s_x)\)
  (does not hold at \(s\) = 1)
\item
  \({}_{s}q_{x + t}\) = 1 - \(p^s_x\)
\end{itemize}

\section{Exercises}\label{exercises}

1.1. Suppose: \(F_0(t)\) = 1 - \((1 + 0.00026t^2)^{-1}\) for \(t\)
\(\ge\) 0.

Calculate the probability that (30) dies between ages 35 and 40.

(A) 0.056 (B) 0.058 (C) 0.060 (D) 0.062 (E) 0.064

1.2. You are given: \(s(x)\) = \(\frac{10,000 - x^2}{10,000}\) for 0
\(\le\) \(x\) \(\le\) 100.

Calculate: \(q_{49}\).

(A) 0.009 (B) 0.011 (C) 0.013 (D) 0.015 (E) 0.017

1.3. Consider a population of newborns (lives aged 0). Each newborn has
mortality such that:

\(S_0(x)\) = \(\frac{x^2}{\omega^2}\) - \(\frac{2x}{\omega}\) + 1 for 0
\(\le\) \(x\) \(\le\) \(\omega\).

It is assumed that \(\omega\) varies among newborns, and is a random
variable with a uniform distribution between ages 85 and 105.

Calculate the probability that a random newborn survives to age 18.

(A) 0.656 (B) 0.657 (C) 0.658 (D) 0.659 (E) 0.660

1.4. Suppose: \(S_0(t)\) = \(\exp[-\frac{t^2}{2500}]\) for \(t\) \(\ge\)
0.

Calculate the force of mortality at age 45.

(A) 0.036 (B) 0.039 (C) 0.042 (D) 0.045 (E) 0.048

1.5. The probability density function of the future lifetime of a brand
new machine is: \(f(x)\) = \(\frac{4x^3}{27c}\) for 0 \(\le\) \(x\)
\(\le\) \(c\).

Calculate: \(\mu(1.1)\).

(A) 0.06 (B) 0.07 (C) 0.08 (D) 0.09 (E) 0.10

1.6. You are given:

(i) The probability that (30) will die within 30 years is 0.10.

(ii) The probability that (40) will survive to at least age 45 and that
another (45) will die by age 60 is 0.077638.

(iii) The probability that two lives age 30 will both die within 10
years is 0.000096.

(iv) All lives are independent and have the same expected mortality.

Calculate the probability that (45) will survive 15 years.

(A) 0.90 (B) 0.91 (C) 0.92 (D) 0.93 (E) 0.94

1.7. You are given:

(i) \(e_{50}\) = 20 and \(e_{52}\) = 19.33

(ii) \(q_{51}\) = 0.035

Calculate: \(q_{50}\).

(A) 0.028 (B) 0.030 (C) 0.032 (D) 0.034 (E) 0.036

1.8. You are given:

\(S_0(x)\) = \(\Big(\frac{1 + 0.005(1.1)^x}{1.005}\Big)\)\(^{-0.2098}\)
for \(x\) \(>\) 0.

Calculate the force of mortality at age 30.

(A) 0.0012 (B) 0.0016 (C) 0.0020 (D) 0.0024 (E) 0.0028

1.9. For a population of smokers and non-smokers:

(i) Non-smokers have a force of mortality that is equal to one-half the
force of mortality for smokers at each age.

(ii) For non-smokers, mortality follows a uniform distribution with
\(\omega\) = 90.

Calculate the difference between the probability that a 55 year old
smoker dies within 10 years and the probability that a 55 year old
non-smoker dies within 10 years.

(A) 0.20 (B) 0.22 (C) 0.24 (D) 0.26 (E) 0.28

1.10. You are given:

(i) The standard probability that (40) will die prior to age 41 is 0.01.

(ii) (40) is now subject to an extra risk during the year between ages
40 and 41.

(iii) To account for the extra risk, a revised force of mortality is
defined for the year between ages 40 and 41.

(iv) The revised force of mortality is equal to the standard force of
mortality plus a term that decreases linearly from 0.05 at age 40 to 0
at age 41.

Calculate the revised probability that (40) will die prior to age 41.

(A) 0.030 (B) 0.032 (C) 0.034 (D) 0.036 (E) 0.038

1.11. An actuary assumes that Jed, aged 40, has the force of mortality:

\(\mu_x\) = \(\frac{x^2}{c^3 - x^3}\) for 0 \(\le\) \(x\) \(<\) \(c\).

Using \(\mu_x\), the actuary calculates the probability that Jed dies
within 20 years as 0.06844. However, \(\mu_x\) is only appropriate for a
life with standard mortality. Jed is actually a substandard life with
force of mortality:

\(\mu^{*}_x\) = 3\(\mu_x\) = \(\frac{3x^2}{c^3 - x^3}\) for 0 \(\le\)
\(x\) \(<\) \(c\).

Using \(\mu^{*}_x\), calculate the correct value of the probability that
Jed dies within 20 years.

(A) 0.16 (B) 0.17 (C) 0.18 (D) 0.19 (E) 0.20

1.12. You are given:

(i) \(T_x\) denotes the time-until-death random variable for (x).

(ii) Mortality follows de Moivre's Law with limiting age \(\omega\).

(iii) The variance of \(T_{25}\) is equal to 352.0833.

Calculate: \(\mathring{e}_{40: \overline{10}|}\).

(A) 7.5 (B) 8.0 (C) 8.5 (D) 9.0 (E) 9.5

1.13. You are given: \(\mu_x\) = \(\frac{1}{\sqrt{80 - x}}\) for 0
\(\le\) \(x\) \(<\) 80.

Calculate the median future lifetime for (40).

(A) 4.0 (B) 4.3 (C) 4.6 (D) 4.9 (E) 5.2

1.14. You are given:

\[\mu_{x} = \left\{
  \begin{array}{l l}
    0.04 & \quad \text{for 0 $\le$ x $<$ 40}\\
    0.05 & \quad \text{for 40 $\le$ x}\\
  \end{array} \right.\]

Calculate: \(\mathring{e}_{25}\).

(A) 22 (B) 23 (C) 24 (D) 25 (E) 26

1.15. You are given: \({}_{k|}q_0\) = 0.10 for \(k\) = 0, 1, \ldots{},
9.

Calculate: \({}_{5}p_2\).

(A) 0.275 (B) 0.325 (C) 0.375 (D) 0.425 (E) 0.475

1.16. For the current model of Zingbot:

(i) \(s(x)\) = \(\frac{\omega - x}{\omega}\) for 0 \(\le\) \(x\) \(\le\)
\(\omega\)

(ii) \(var[T(5)]\) = 102.083333.

For the proposed model of Zingbot, with the same \(\omega\) as the
current model:

(1) \(s^*(x)\) = \((\frac{\omega - x}{\omega})^{\alpha}\) for 0 \(\le\)
\(x\) \(\le\) \(\omega\), \(\alpha\) \(>\) 0

(2) \(\mu^*_{10}\) = 0.0166667.

Calculate the difference between the complete expectation of life for a
brand new proposed model of Zingbot and the complete expectation of life
for a brand new current model of Zingbot.

(A) 5.9 (B) 6.1 (C) 6.3 (D) 6.5 (E) 6.7

1.17. Mortality for Frodo, age 33, is usually such that:

\({}_{t}p_x\) = \(\left(\frac{110 - x - t}{110 - x}\right)^{2}\) for 0
\(\le\) \(t\) \(\le\) \(110 - x\).

However, Frodo has decided to embark on a dangerous quest that will last
for the next three years (starting today). During these three years
only, Frodo's mortality will be revised so that he will have a constant
force of mortality of 0.2 for each year. After the quest, Frodo's
mortality will once again follow the above expression for \({}_{t}p_x\).

Calculate Frodo's revised complete expectation of life.

(A) 15.2 (B) 15.4 (C) 15.6 (D) 15.8 (E) 16.0

1.18. You are given:

\[\mu_x = \left\{
  \begin{array}{l l l}
    0.020       & \quad \text{for 20 $\le$ $x$ $<$ 30}\\
    0.025       & \quad \text{for 30 $\le$ $x$ $<$ 42}\\
    0.030       & \quad \text{for 42 $\le$ $x$ $<$ 60}\\
  \end{array} \right.\]

Calculate the probability that (26) dies in the 19th year.

(A) 0.015 (B) 0.017 (C) 0.019 (D) 0.021 (E) 0.023

1.19. An actuary has developed a survival model for a widget, denoted by
A, such that:

\(S^A_0(x)\) = \(\frac{(10 - x)^2}{100}\) for 0 \(\le\) \(x\) \(\le\)
10.

The actuary's supervisor notes that the above survival model is
incorrect. The correct survival model for a widget, denoted by B, is
such that:

\[S^B_0(x) = \left\{
  \begin{array}{l l }
    \frac{(10 - x)^2}{100}       & \quad \text{for 0 $\le$ $x$ $<$
    5} \vspace{5mm}\\
  e^{- 0.2\ln(4) x}            & \quad \text{for x $\ge$ 5}\\
  \end{array} \right.\]

Calculate: \(\mathring{e}^B_2\) - \(\mathring{e}^A_2\).

(A) 0.70 (B) 0.72 (C) 0.74 (D) 0.76 (E) 0.78

1.20. For a group of lives aged 40, consisting of 30\% smokers and 70\%
non-smokers, you are given:

(i) For non-smokers, \(\mu^N(x)\) = 0.05 for \(x\) \(\ge\) 40.

(ii) For smokers, \(\mu^S(x)\) = 0.10 for \(x\) \(\ge\) 40.

Calculate the 90th percentile of the distribution of the future lifetime
of a randomly selected member from this population.

(A) 40 (B) 42 (C) 44 (D) 46 (E) 48

1.21. You are given:

(i) \(T_x\) is the time-until-death for (x) random variable.

(ii) The force of mortality is constant.

(iii) \(e_x\) = 15.63

Calculate the variance of \(T_x\).

(A) 240 (B) 250 (C) 260 (D) 270 (E) 280

1.22. Originally, mortality for Daniel, currently aged 30, is such that:

(i) \(e_{30}\) = 40.78

(ii) \(e_{30 :\overline{15}|}\) = 14.07

(iii) \({}_{15}p_{30}\) = 0.8764 and \({}_{16}p_{30}\) = 0.8664

(iv) The limiting age is 100.

Now, it is believed that in the year of age between ages 45 and 46,
Daniel will be subject to an additional risk such that the constant 0.15
will be added to the force of mortality \(\mu_{45}(t)\) for 0 \(\le\)
\(t\) \(<\) 1.

Calculate the revised value of \(e_{30}\) for Daniel, accounting for the
additional risk in the year of age between ages 45 and 46.

(A) 36 (B) 37 (C) 38 (D) 39 (E) 40

1.23. For (x):

(i) \(K\) is the curtate future lifetime random variable.

(ii) \[{}_{k|}q_{x} = \left\{
  \begin{array}{l l}
    0.20       & \quad \text{for $k$ = 0, 1, 2}\\
    0.40       & \quad \text{for $k$ = 3 }\\
  \end{array} \right.\]

Calculate the standard deviation of \(K\).

(A) 1.1 (B) 1.2 (C) 1.3 (D) 1.4 (E) 1.5

1.24. You are given:

(i) \(\mu(x)\) = \(B\)\((1.05)^{x}\) for \(x\) \(\ge\) 0, \(B\) \(>\) 0.

(ii) \(p_{51}\) = 0.9877

Calculate: \(B\).

(A) 0.001 (B) 0.002 (C) 0.003 (D) 0.004 (E) 0.005

1.25. You are given:

(i) The force of mortality for Vivian is \(\mu^{V}_x\) = \(\mu\) for
\(x\) \(\ge\) 0, \(\mu\) \(>\) 0.

(ii) The force of mortality for Augustine is \(\mu^{A}_x\) =
\(\frac{1}{90 - x}\) for 0 \(\le\) \(x\) \(<\) 90.

Calculate \(\mu\) so that \({}_{10}p_{30}\) is the same for Vivian and
Augustine.

(A) 0.016 (B) 0.018 (C) 0.020 (D) 0.022 (E) 0.024

1.26. Consider the following survival function:

\(S_0(x)\) = \(0.0125\sqrt{k^2 - x^2}\) for 0 \(\le\) \(x\) \(\le\)
\(k\).

Calculate the force of mortality at age 46.

(A) 0.011 (B) 0.013 (C) 0.015 (D) 0.017 (E) 0.019

1.27. For a certain model of widget, the Widget T-1000:

\(\mu_x\) = \(\frac{2x}{c^2 - x^2}\) for 0 \(\le\) \(x\) \(<\) \(c\).

A brand new Widget T-1000 has a complete expectation of life equal to 8
years. Calculate the probability that a one-year old Widget T-1000
survives at least two years but no more than four years.

(A) 0.11 (B) 0.12 (C) 0.13 (D) 0.14 (E) 0.15

1.28. Actuary A and Actuary B are each trying to calculate the 3-year
temporary curtate life expectancy for Miguel, aged 60.

Both agree that a standard life has the following force of mortality:

\(\mu^S_x\) = 0.00006 + 0.0000006\((1.15)^{x}\) for \(x\) \(\ge\) 0.

Both also agree that Miguel has a force of mortality that exceeds the
force of mortality for a standard life at all ages. However, Actuary A
believes that Miguel has force of mortality 1.15\(\mu^S_x\), while
Actuary B believes that Miguel has force of mortality \(\mu^S_x\) +
1.15.

Calculate the difference between Actuary A's 3-year temporary curtate
life expectancy for Miguel and Actuary B's 3-year temporary curtate life
expectancy for Miguel.

(A) 2.1 (B) 2.3 (C) 2.5 (D) 2.7 (E) 2.9

1.29. For a population of lives each aged 55 that consists of smokers
and non-smokers:

(i) For smokers, \(\mu_x\) = 0.0008\(x\) for \(x\) \(\ge\) 0.

(ii) The force of mortality for a smoker is twice the force of mortality
for a non-smoker at each age.

Calculate the median future lifetime for a non-smoker from this
population.

(A) 24.0 (B) 24.5 (C) 25.0 (D) 25.5 (E) 26.0

1.30. Suppose today is January 1, 2014, and Paul has just turned age 35.
He has mortality such that:

\({}_{t}p_{35}\) = \((0.95)^t\) for \(t\) \(\ge\) 0.

Calculate the probability that Paul will die in an odd-numbered calendar
year.

(A) 0.48 (B) 0.49 (C) 0.50 (D) 0.51 (E) 0.52

1.31. Consider the following life table, where missing entries are
denoted by ``---'':

\begin{longtable}[]{@{}rrrr@{}}
\toprule
\(\mathbf{x}\) & \(\mathbf{q_x}\) & \(\mathbf{l_x}\) &
\(\mathbf{d_x}\)\tabularnewline
\midrule
\endhead
48 & --- & 90,522 & ---\tabularnewline
49 & 0.007453 & 89,900.9286 & ---\tabularnewline
\bottomrule
\end{longtable}

Calculate the expected number of deaths between ages 48 and 50.

(A) 1280 (B) 1290 (C) 1300 (D) 1310 (E) 1320

1.32. You are given the following life table, where missing entries are
denoted by ``---'':

\begin{longtable}[]{@{}rrrr@{}}
\toprule
\(\mathbf{x}\) & \(\mathbf{l_{x}}\) & \(\mathbf{q_{x}}\) &
\(\mathbf{e_{x}}\)\tabularnewline
\midrule
\endhead
65 & 79,354 & 0.0172 & ---\tabularnewline
66 & --- & 0.0186 & ---\tabularnewline
67 & --- & --- & ---\tabularnewline
68 & 74,993 & --- & 14.89\tabularnewline
69 & --- & --- & 14.22\tabularnewline
\bottomrule
\end{longtable}

Calculate the expected number of deaths between ages 67 and 69.

(A) 2970 (B) 3020 (C) 3070 (D) 3120 (E) 3170

1.33. You are given:

(i) \(l_x\) = 1000(\(\omega\) - \(x\)) for 0 \(\le\) \(x\) \(\le\)
\(\omega\)

(ii) \(\mu_{30}\) = 0.0125

Calculate: \(\mathring{e}_{40 :\overline{20}|}\).

(A) 17.1 (B) 17.6 (C) 18.1 (D) 18.6 (E) 19.1

1.34. You are given the following life table, where missing values are
indicated by ``---'':

\begin{longtable}[]{@{}rrrr@{}}
\toprule
\(\mathbf{x}\) & \(\mathbf{l_x}\) & \(\mathbf{d_x}\) &
\(\mathbf{p_x}\)\tabularnewline
\midrule
\endhead
0 & 1000.0 & --- & 0.875\tabularnewline
1 & --- & 125.0 & ---\tabularnewline
2 & --- & --- & ---\tabularnewline
3 & --- & --- & 0.680\tabularnewline
4 & --- & 182.5 & ---\tabularnewline
5 & 200.0 & --- & ---\tabularnewline
\bottomrule
\end{longtable}

Calculate: \({}_{2|}q_{0}\).

(A) 0.16 (B) 0.17 (C) 0.18 (D) 0.19 (E) 0.20

1.35. Woolhouse is currently age 40. Woolhouse's mortality follows 130\%
of the Illustrative Life Table; that is, \(q_x\) for Woolhouse is 130\%
of \(q_x\) in the Illustrative Life Table for x = 40, 41, \ldots{}, 110.

Calculate Woolhouse's 4-year temporary curtate life expectancy.

(A) 3.950 (B) 3.955 (C) 3.960 (D) 3.965 (E) 3.970

1.36. Suppose mortality follows the Illustrative Life Table, and deaths
are uniformly distributed within each year of age.

Calculate: \({}_{4.5}q_{40.3}\).

(A) 0.0141 (B) 0.0142 (C) 0.0143 (D) 0.0144 (E) 0.0145

1.37. Suppose mortality follows the Illustrative Life Table, where
deaths are assumed to be uniformly distributed between integer ages.

Calculate the median future lifetime for (32).

(A) 44.7 (B) 45.0 (C) 45.3 (D) 45.6 (E) 45.9

1.38. Suppose mortality follows the Illustrative Life Table with the
assumption that deaths are uniformly distributed between integer ages.

Calculate: \({}_{0.9}q_{60.6}\).

(A) 0.0130 (B) 0.0131 (C) 0.0132 (D) 0.0133 (E) 0.0134

1.39. You are given the mortality rates:

\(q_{30}\) = 0.020, \(q_{31}\) = 0.019, \(q_{32}\) = 0.018.

Assume deaths are uniformly distributed over each year of age.

Calculate the 1.4-year temporary complete life expectancy for (30).

(A) 1.377 (B) 1.379 (C) 1.381 (D) 1.383 (E) 1.385

1.40. Using the Illustrative Life Table, calculate:
\({}_{11|17}q_{42}\).

(A) 0.20 (B) 0.21 (C) 0.23 (D) 0.24 (E) 0.25

1.41. Consider two survival models A and B:

(i) For Model A: \(l_x\) = 1000(\(\omega_A\) - \(x\)) for 0 \(\le\)
\(x\) \(\le\) \(\omega_A\)

(ii) For Model B: \(l_x\) = 500\((\omega_B - x)^{\alpha}\) for 0 \(\le\)
\(x\) \(\le\) \(\omega_B\), \(\alpha\) \(>\) 0

Furthermore:

(1) For Model B, the force of mortality at age 55 is 0.046.

(2) The complete expectation of life for (40) under Model A is 39.615\%
higher than the complete expectation of life for (40) under Model B.

(3) For Model A, the probability that (45) survives the first 20 years
and dies in the subsequent 10 years is 0.20.

For Model B, calculate the probability that (45) dies between ages 65
and 75.

(A) 0.16 (B) 0.17 (C) 0.18 (D) 0.19 (E) 0.20

1.42. Consider a population that consists of 600 lives aged 50 and 520
lives aged 60.

Each life has mortality that follows the Illustrative Life Table, and
all lives have independent future lifetime random variables.

Calculate the standard deviation of the total number of survivors to age
80.

(A) 14.7 (B) 15.2 (C) 15.7 (D) 16.2 (E) 16.7

1.43. Suppose:

(i) \(q_{70}\) = 0.04 and \(q_{71}\) = 0.05.

(ii) Let UDD denote a uniform distribution of deaths assumption within
each year of age, and let CF denote a constant force of mortality within
each year of age.

Calculate the probability that (70.6) will die within the next 0.5 years
under UDD minus the probability that (70.6) will die within the next 0.5
years under CF.

(A) 0.00008 (B) 0.00010 (C) 0.00012 (D) 0.00014 (E) 0.00016

1.44. You are given:

(i) The force of mortality is constant between integer ages.

(ii) \({}_{0.3}q_{x + 0.7}\) = 0.10

Calculate: \(q_x\).

(A) 0.24 (B) 0.26 (C) 0.28 (D) 0.30 (E) 0.32

1.45. A life insurer issues Roderick, aged 40, a policy that will pay
10,000 upon survival of a number of years equal to Roderick's median
future lifetime. You are given:

(i) \(d\) = 0.04

(ii) For Roderick: \(q_{40 + k}\) = 0.05(1 + \(k\)) for \(k\) = 0, 1,
\ldots{}, 19. (Roderick is a very unfortunate individual with respect to
his future lifetime distribution.)

(iii) The force of mortality is constant between integer ages.

Calculate the expected present value of the 10,000 payment; that is,
calculate the present value of 10,000 times the probability that the
10,000 will be paid to Roderick.

(A) 4100 (B) 4130 (C) 4160 (D) 4190 (E) 4220

1.46. For a mortality table:

(i) \(q_{70}\) = 0.058

(ii) Deaths are uniformly distributed within each year of age.

(iii) The probability that (71.25) dies within 0.4 years is 0.0252.

Calculate the probability that (70) dies between ages 71.25 and 71.65.

(A) 0.021 (B) 0.023 (C) 0.025 (D) 0.027 (E) 0.029

1.47. Andy, aged 66, has mortality rates that are 3 times higher than
mortality rates in the Illustrative Life Table. That is, for \(x\) = 66,
67, \ldots{}, 110:

\(q^{*}_x\) = 3\(q_x\),

where \(q_x\) is a mortality rate in the Illustrative Life Table and
\(q^{*}_x\) is Andy's corresponding mortality rate.

Calculate the probability that Andy dies between ages 68 and 70.

(A) 0.11 (B) 0.12 (C) 0.13 (D) 0.14 (E) 0.15

\subsection{Answers to Exercises}\label{answers-to-exercises}

1.1. E 1.26. A

1.2. C 1.27. A

1.3. A 1.28. C

1.4. A 1.29. D

1.5. B 1.30. B

1.6. C 1.31. B

1.7. B 1.32. E

1.8. B 1.33. A

1.9. A 1.34. D

1.10. C 1.35. C

1.11. D 1.36. E

1.12. D 1.37. C

1.13. B 1.38. A

1.14. A 1.39. C

1.15. C 1.40. C

1.16. E 1.41. D

1.17. D 1.42. E

1.18. C 1.43. A

1.19. D 1.44. D

1.20. A 1.45. C

1.21. C 1.46. B

1.22. B 1.47. D

1.23. B

1.24. A

1.25. B

\section{Past Exam Questions}\label{past-exam-questions}

\begin{itemize}
\item
  Exam MLC, Fall 2015: \#1, 2
\item
  Exam MLC, Spring 2015: \#1
\item
  Exam MLC, Spring 2014: \#1
\item
  Exam MLC, Fall 2013: \#24, 25
\item
  Exam 3L, Fall 2013: \#1, 2, 3
\item
  Exam MLC, Spring 2013: \#20
\item
  Exam 3L, Spring 2013: \#1, 2, 3
\item
  Exam MLC, Fall 2012: \#3
\item
  Exam 3L, Fall 2012: \#1, 2, 3
\item
  Exam MLC, Spring 2012: \#2 (MLC Only)
\item
  Exam 3L, Spring 2012: \#1, 2, 3
\item
  Exam MLC, Sample Questions: \#13, 21, 22, 28, 32, 59, 65, 98, 106,
  116, 120, 131, 145, 155, 161, 171, 188, 189, 200, 201, 207, 219, 223,
  267 (MLC Only), 276
\item
  Exam 3L, Fall 2011: \#1, 2
\item
  Exam 3L, Spring 2011: \#1, 2, 3
\item
  Exam 3L, Fall 2010: \#1, 2, 3
\item
  Exam 3L, Spring 2010: \#1, 2, 3, 4
\item
  Exam 3L, Fall 2009: \#1, 2, 3
\item
  Exam 3L, Spring 2009: \#1, 3
\item
  Exam 3L, Fall 2008: \#12, 13, 14
\item
  Exam 3L, Spring 2008: \#13, 14, 15, 16
\item
  Exam MLC, Spring 2007: \#1, 21
\end{itemize}

\chapter{Selection}\label{selection}

\section{Key Concepts}\label{key-concepts}

For a life table based on an insured population, one must consider for
each individual both (i) the age of policy issue and (ii) the time that
has elapsed since policy issue. This is because the insurer typically
underwrites individuals that purchase a policy. Through the
underwriting, the insurer learns additional information about the
individual's survival distribution that the insurer would not know for a
life randomly drawn from the general population. This additional
information must be accounted for in the calculation of various
quantities such as survival probabilities for the individual and the
value of the individual's policy.

\begin{itemize}
\item
  Consider an individual now aged x + t who purchased a policy at age x.
  We say that the individual was \textbf{selected}, or \textbf{select},
  at age x (and time \(t\) = 0).
\item
  The additional information gained from underwriting the above
  individual, obtained by surveys and/or a medical examination, is
  assumed to apply for a certain number of years after policy issue
  called the \textbf{select period}.
\item
  Say the select period is \(d\) years. For \(t\) \(<\) \(d\), one
  accounts for the initial selection of the above individual at age x;
  the individual's current age would be written as \[x\] + t (the select
  brackets \[ \] denote the initial age of selection). For \(t\) \(\ge\)
  \(d\), one no longer accounts for the initial selection of the
  individual at age x, and the individual's age would be written simply
  as x + t (with no select brackets \[ \]).
\item
  An individual has \textbf{select mortality} for ages/times within the
  select period that differs from the mortality of the general
  population. An individual has \textbf{ultimate mortality} for
  ages/times beyond the select period where their mortality is assumed
  to be the same as a life from the general population.
\item
  A life table that accounts for both select and ultimate mortality is
  called a \textbf{select-and-ultimate life table}.
\item
  A life table that ignores selection completely is called an
  \textbf{aggregate life table}.
\item
  The previous formulas for the quantities considered so far, such as
  survival probabilities, are still valid in the event of selection. One
  simply has to use information from the select part of the
  select-and-ultimate life table for ages/times within the select
  period.
\item
  For example, with a select period of 3 years, \({}_{2}p_{[x]}\) =
  \((p_{[x]})\)\((p_{[x] + 1})\) and \({}_{5}p_{[x]}\) =
  \((p_{[x]})\)\((p_{[x] + 1)}\)\((p_{[x] + 2})\)\((p_{x + 3})\)\((p_{x + 4})\).
  The p's with select brackets would come from the select part of the
  select-and-ultimate life table, and the p's without select brackets
  would come from the ultimate part.
\item
  An illustrative select-and-ultimate table is the

  \textbf{Standard Select and Ultimate Survival Model}. This table is
  provided in Appendix D of Dickson et al. For brevity, I will refer to
  this table as the \textbf{Standard Select Survival Model}.
\end{itemize}

\section{Exercises}\label{exercises-1}

2.1. Mortality follows the select-and-ultimate life table:

Calculate: 10,000\({}_{1|}q_{[30]}\).

(A) 1 (B) 2 (C) 3 (D) 4 (E) 5

2.2. Suppose mortality follows the Standard Select Survival Model.

Calculate: \({}_{1|2}q_{[70]+1}\).

(A) 0.025 (B) 0.027 (C) 0.029 (D) 0.031 (E) 0.033

2.3. Consider the following select-and-ultimate life table:

\begin{longtable}[]{@{}rrrrrr@{}}
\toprule
\(\mathbf{x}\) & \(\mathbf{q_{[x]}}\) & \(\mathbf{q_{[x] + 1}}\) &
\(\mathbf{q_{[x] + 2}}\) & \(\mathbf{q_{x + 3}}\) &
\(\mathbf{x + 3}\)\tabularnewline
\midrule
\endhead
60 & 0.09 & 0.11 & 0.13 & 0.15 & 63\tabularnewline
61 & 0.10 & 0.12 & 0.14 & 0.16 & 64\tabularnewline
62 & 0.11 & 0.13 & 0.15 & 0.17 & 65\tabularnewline
63 & 0.12 & 0.14 & 0.16 & 0.18 & 66\tabularnewline
64 & 0.13 & 0.15 & 0.17 & 0.19 & 67\tabularnewline
\bottomrule
\end{longtable}

Assume that deaths follow the uniform distribution of deaths assumption
between integer ages.

Calculate: \({}_{1.6}q_{[61] + 0.75}\).

(A) 0.1855 (B) 0.1856 (C) 0.1857 (D) 0.1858 (E) 0.1859

2.4. A select-and-ultimate life table with a select period of 2 years is
based on probabilities that satisfy the following relationship:

\(q_{[x - i] + i}\) = \(\frac{3}{5 - i} \times q_x\) for \(i\) = 0, 1.

You are given that \(l_{68}\) = 10,000, \(q_{66}\) = 0.026, and
\(q_{67}\) = 0.028.

Calculate: \(l_{[65] + 1}\).

(A) 10,414 (B) 10,451 (C) 10,479 (D) 10,493 (E) 11,069

2.5. Suppose mortality follows the Standard Select Survival Model.

Calculate: \(e_{[60] :\overline{5}|}\).

(A) 4.928 (B) 4.932 (C) 4.936 (D) 4.940 (E) 4.944

2.6. Consider a select-and-ultimate life table with a 2-year select
period.

You are given:

(i) \(l_{[35]}\) = 1500

(ii) \(l_{36}\) = 1472.31

(iii) \(q_{[35]}\) = 0.0240

(iv) \(q_{[35] + 1}\) = 0.0255

Calculate: \(l_{35}({}_{1|}q_{35})\).

(A) 42 (B) 44 (C) 46 (D) 48 (E) 50

2.7. Quinn is currently age 60. He was selected by the PlzDntDie Life
Insurance Company one year ago. Quinn has mortality that follows a
select-and-ultimate life table with a 2-year select period:

(i) The ultimate part of the model is such that:

\(\mu_x\) = 0.0002\((1.1)^x\) for \(x\) \(\ge\) 0.

(ii) The select part of the model is such that:

\(\mu_{[x] + s}\) = \((0.8)^{2 - s}\mu_{x + s}\) for \(x\) \(\ge\) 0, 0
\(\le\) \(s\) \(\le\) 2.

Calculate the probability that Quinn survives to age 61.

(A) 0.938 (B) 0.944 (C) 0.950 (D) 0.956 (E) 0.962

2.8. You are given:

(i) Mortality follows the Standard Select Survival Model.

(ii) Deaths are uniformly distributed over each year of age.

Calculate: \(\mathring{e}_{[75] + 1 :\overline{1.3}|}\).

(A) 1.277 (B) 1.280 (C) 1.283 (D) 1.287 (E) 1.290

2.9. For a select and ultimate life table with a 1-year select period:

(i) \(\mu_{[55] + t}\) = 0.5\(\mu_{55 + t}\) for 0 \(\le\) \(t\) \(\le\)
1

(ii) \(e_{55}\) = 18.02

(iii) \(e_{[55]}\) = 18.33

Calculate: \(e_{56}\).

(A) 17.60 (B) 17.65 (C) 17.70 (D) 17.75 (E) 17.80

2.10. Consider a population of lives each age 55 and selected at that
age, where 70\% are non-smokers and 30\% are smokers.

The force of mortality is:

\(\mu_{[55] + t}\) = 0.009\(t\)(0.4 + 0.1\(S\)) for \(t\) \(\ge\) 0,

where \(S\) = 0 for a non-smoker and \(S\) = 1 for a smoker.

Calculate the probability that a randomly chosen life from the above
population will die before age 75.

(A) 0.51 (B) 0.52 (C) 0.53 (D) 0.54 (E) 0.55

2.11. For a select-and-ultimate life table with a 2-year select period:

(i) Ultimate mortality follows the Illustrative Life Table.

(ii) \(q_{[x]}\) = 0.5\(q_x\), where \(q_x\) is from the Illustrative
Life Table.

(iii) \(q_{[x]+1}\) = 0.25\(q_{x + 1}\), where \(q_{x + 1}\) is from the
Illustrative Life Table.

Calculate the probability that a select life aged 70 will die after age
75.

(A) 0.79 (B) 0.81 (C) 0.83 (D) 0.85 (E) 0.87

2.12. Consider the setup in Exercise 2.11. Calculate the 4-year
temporary curtate expectation of life for a select life aged 70.

(A) 3.79 (B) 3.81 (C) 3.83 (D) 3.85 (E) 3.87

\subsection{Answers to Exercises}\label{answers-to-exercises-1}

2.1. C

2.2. B

2.3. C

2.4. D

2.5. E

2.6. C

2.7. B

2.8. C

2.9. B

2.10. D

2.11. D

2.12. A

\section{Past Exam Questions}\label{past-exam-questions-1}

\begin{itemize}
\item
  Exam MLC, Fall 2014: \#20
\item
  Exam MLC, Spring 2014: \#2
\item
  Exam MLC, Fall 2013: \#3
\item
  Exam MLC, Spring 2013: \#19
\item
  Exam MLC, Fall 2012: \#2
\item
  Exam MLC, Spring 2012: \#1, 13
\item
  Exam MLC, Sample Questions: \#66, 73, 136, 168
\item
  Exam MLC, Spring 2007: \#18
\end{itemize}

\chapter{Annuities I}\label{annuities-i}

\section{Key Concepts}\label{key-concepts-1}

A \textbf{life annuity} policy provides payments to an
\textbf{annuitant} each period while that person survives.

Annuities can be described as either (i) \textbf{continuous}: the
payments are made continuously each year while the annuitant survives or
(ii) \textbf{discrete}: the payments are made at the beginning or the
end of each period while the annuitant survives. If the payments are
made at the beginning of each period, the policy is an
\textbf{annuity-due}; if the payments are made at the end of each
period, the policy is an \textbf{annuity-immediate}.

\textbf{Types of Life Annuities}:

\begin{itemize}
\item
  \textbf{Whole Life Annuity:} Provides payments each period while the
  annuitant survives.
\item
  \textbf{Temporary Life Annuity:} Provides payments each period while
  the annuitant survives for at most \(n\) years after policy issue.
  This is also called a \textbf{term annuity}.
\item
  \textbf{Deferred Life Annuity:} The annuitant must survive a
  \(u\)-year deferral period after policy issue in order for any
  payments to be made. A \textbf{deferred whole life annuity} provides
  payments each period while the annuitant survives if the annuitant
  first survives the \(u\)-year deferral period; that is, payments are
  made after \(u\) years while the annuitant survives. A
  \textbf{deferred temporary life annuity} provides payments each period
  while the annuitant survives for at most \(n\) years after first
  surviving the \(u\)-year deferral period; that is, payments are made
  between \(u\) years and \(u + n\) years after policy issue while the
  annuitant survives.
\item
  \textbf{Certain and Life Annuity:} Is guaranteed to provide payments
  for the first \(n\) years after policy issue, regardless of whether
  the annuitant survives or dies within the \(n\)-year period. If the
  annuitant survives the first \(n\) years, the annuity continues to
  provide payments each period for as long as the annuitant survives.
  This is also called a \textbf{guaranteed annuity}.

  For each of these annuities, the following will be considered:

  \begin{itemize}
  \item
    The \textbf{present value of the annuity}, \(Y\), are the payments
    discounted for interest between policy issue and each payment date.
    This is a random variable, as the number of payments is a function
    of the future lifetime of the annuitant.
  \item
    The \textbf{expected present value of the annuity}, \(E(Y)\), are
    the payments discounted for both interest and survival between
    policy issue and each potential payment date. \(E(Y)\) is with
    respect to the distribution of the annuitant's future lifetime.

    \(E(Y)\) will be written differently for each type of life annuity
    considered. In addition, \(E(Y)\) can also be called the
    \textbf{actuarial present value} \textbf{of the annuity}, the
    \textbf{single premium}, the \textbf{net single premium}, or the
    \textbf{single benefit premium}.
  \item
    Secondary characteristics of the distribution of \(Y\) that will be
    of interest include the variance of \(Y\) and percentiles of the
    distribution of \(Y\).
  \end{itemize}
\end{itemize}

\section{Life Annuity Formulas}\label{life-annuity-formulas}

This section provides key formulas for different life annuities. Note:

\begin{itemize}
\item
  Life annuities have continuous and discrete versions. In the
  continuous case, the payments are made continuously each year up until
  the moment of death of the annuitant (\(T_x\) years after policy
  issue). The discrete case can be : (i) annual or (ii) \(m\)-thly (Exam
  MLC only). In (i), the payments are made at either the beginning of
  each year (for a total of \(K_x\) + 1 payments), or payments are made
  at the end of each year (for a total of \(K_x\) payments).

  It is possible that the number of payments made could be subject to a
  finite term and/or a deferral period. It is also possible that there
  may be a guaranteed number of payments.
\item
  The general formula for the present value of a continuous life annuity
  on (x) with payment rate \(\pi_t\) at time \(t\) (\(>\) 0) is:

  \(Y\) = \(\int^{T_x}_0 \pi_t v^{t}dt\) =
  \(\int^{T_x}_0 \pi_t e^{-\delta t}dt\).

  The general formula for the expected present value of a continuous
  life annuity on (x) with payment rate \(\pi_t\) at time \(t\) (\(>\)
  0) is:

  \(E(Y)\) = \(\int^{\infty}_0 \pi_t v^{t}{}_{t}p_xdt\) =
  \(\int^{\infty}_0 \pi_t e^{-\delta t}{}_{t}p_xdt\) =
  \(\int^{\infty}_0 \pi_t ({}_{t}E_x)dt\). Furthermore:

  ``Say the payment rate is \(\pi_t\) at time \(t\). Then the present
  value of this benefit at time \(t\) is \(\pi_t v^{t}dt\), and the
  expected present value of this benefit is \(\pi_t v^{t}{}_{t}p_xdt\)
  ((x) has to survive to time \(t\) in order for \(\pi_tdt\) to be made
  at that time). Integrating over all possible payment times provides
  the overall expected present value. This is the \textbf{current
  payment approach.}''
\end{itemize}

\begin{itemize}
\item
  The general formula for the present value of an annual life
  annuity-due that pays \(\pi_{k}\) at time \(k\) (\(k\) = 0, 1, 2,
  \ldots{}) is:

  \(Y\) = \(\sum^{K_x}_{k = 0} \pi_{k} v^{k}\).

  The general formula for the expected present value of an annual life
  annuity-due that pays \(\pi_{k}\) at time \(k\) (\(k\) = 0, 1, 2,
  \ldots{}) is:

  \(E(Y)\) = \(\sum^{\infty}_{k = 0} \pi_{k} v^{k}{}_{k}p_x\) =
  \(\sum^{\infty}_{k = 0} \pi_{k} ({}_{k}E_x)\). Furthermore:

  ``Say the payment is \(\pi_k\) at time \(k\). Then the present value
  of this benefit is \(\pi_{k} v^{k}\), and the expected present value
  of this benefit is \(\pi_{k} v^{k}{}_{k}p_x\). Summing over all
  possible payment times provides the overall expected present value.
  This is the current payment approach.''

  Note: The general formulas for the present value and expected present
  value of an annual life annuity-immediate on (x) are similar, expect
  there would be no payment at \(k\) = 0.
\item
  The International Actuarial Notation for \(E(Y)\) often contains an
  \(a\), which indicates that the expected present value is for a life
  annuity. For example, \(\ddot{a}_x\) denotes the expected present
  value of a whole life annuity-due of 1 per year on (x).
\item
  \(^2a\) is \(a\) evaluated at double the force of interest.
\item
  \(^2a\) is NOT equal to \(E(Y^2)\). This implies that \(Var(Y)\) is
  NOT equal to \(^2a\) - \(a^2\) (which will usually be negative).
\item
  For each annuity, a (total) payment rate of 1 per year is assumed. If
  the payment rate is \(R\), notation and formulas are adjusted. For
  example, the expected present value of a continuous whole life annuity
  of \(R\) per year on (x) is \(E(Y)\) = \(R\bar{a}_{x}\) =
  \(\int^{\infty}_0 R({}_{t}E_x)\).
\item
  Recursion formulas take an expected present value and decomposes it
  into the sum of two expected present values: the expected present
  value of the annuity during the first period plus the expected present
  value of the annuity if the annuitant survives the first period.

  For example: \(\ddot{a}_x\) = 1 + \(vp_x\ddot{a}_{x + 1}\). ``The
  right-hand side breaks up the expected present value of a whole life
  annuity-due of 1 per year on (x) into the very first payment of 1 at
  issue plus the expected present value of all of the remaining payments
  of 1 at issue.''
\item
  For many annuities, \(Var(Y)\) is too difficult to calculate.
  Therefore, \(Var(Y)\) is omitted for several of the following
  annuities.
\item
  The \textbf{actuarial accumulated value} is the expected present value
  of a payment or payments divided by a discount factor. For example:

  \(\ddot{s}_{x :\overline{n}|}\) =
  \(\frac{1}{{}_{n}E_x}\)\(\ddot{a}_{x:\overline{n}|}\)

  = the actuarial accumulated value at time \(n\) of an \(n\)-year
  temporary life annuity-due of 1 per year on (x).
\end{itemize}

\section{Level Annuities}\label{level-annuities}

\subsection{Whole Life Annuity of 1 per Year on
(x)}\label{whole-life-annuity-of-1-per-year-on-x}

\textbf{Continuous Whole Life Annuity}:

\begin{itemize}
\item
  \(Y\) = \(\bar{a}_{\overline{T_x}|}\) = \(\frac{1 - v^{T_x}}{\delta}\)

  So: \(Y\) = \(\frac{1 - Z}{\delta}\), where \(Z\) is the present value
  random variable for a continuous whole life insurance of 1 on (x).
\item
  \(E(Y)\) = \(\bar{a}_x\) = \(\int^{\infty}_{0} {}_{t}E_xdt\) =
  \(\frac{1 - \bar{A}_x}{\delta}\)

  \begin{itemize}
  \tightlist
  \item
    With a constant force of mortality: \(\bar{a}_x\) =
    \(\frac{1}{\mu + \delta}\)
  \end{itemize}
\item
  It is also true that: \(\bar{a}_x\) =
  \(\int^{\infty}_{0} (\bar{a}_{\overline{t}|}){}_{t}p_x\mu_{x + t} dt\)

  ``Say the moment of death of (x) occurs at time \(t\). Then the
  present value of the payments is \(\bar{a}_{\overline{t}|}\), and the
  expected present value of the payments is
  \((\bar{a}_{\overline{t}|}){}_{t}p_x\mu_{x + t} dt\) ((x) has to
  survive \(t\) years and then immediately die for the present value to
  be \(\bar{a}_{\overline{t}|}\)). Integrating over all times of death
  provides the overall expected present value.''
\item
  \(Var(Y)\) = \(\frac{^2\bar{A}_x - [\bar{A}_x]^2}{\delta^2}\)
\item
  \(F_Y(y)\) = \(Pr[Y \le y]\) = \(Pr[\bar{a}_{\overline{T_x}|} \le y]\)
  = \({}_{c}q_x\) where \(c\) = -\(\frac{\ln(1 - \delta y)}{\delta}\)

  \begin{itemize}
  \item
    For de Moivre's Law (Uniform Distribution):

    \(F_Y(y)\) = -\(\frac{\ln(1 - \delta y)}{\delta (\omega - x)}\) for
    0 \(\le\) \(y\) \(\le\) \(\bar{a}_{\overline{\omega - x}|}\). If the
    annual payment is \(R\), replace \(y\) with \(\frac{y}{R}\).
  \item
    Constant force of mortality: \(F_Y(y)\) = 1 -
    \((1 - \delta y)^\frac{\mu}{\delta}\) for 0 \(\le\) \(y\) \(\le\)
    \(\frac{1}{\delta}\). If the annual payment is \(R\), replace \(y\)
    with \(\frac{y}{R}\).
  \end{itemize}

  Note: The 100\(\alpha\)-th percentile of the distribution of \(Y\),
  \(y_{\alpha}\), solves:

  \(F_Y(y_{\alpha})\) = \(\alpha\) for 0 \(\le\) \(\alpha\) \(\le\) 1.
\end{itemize}

\textbf{Annual Whole Life Annuity}:

\textbf{Whole Life Annuity-Due}:

\begin{itemize}
\item
  \(Y_d\) = \(\ddot{a}_{\overline{K_x + 1}|}\) =
  \(\frac{1 - v^{K_x + 1}}{d}\)

  So: \(Y_d\) = \(\frac{1 - Z}{d}\), where \(Z\) is the present value
  random variable for an annual whole life insurance of 1 on (x).
\item
  \(E(Y_d)\) = \(\ddot{a}_x\) = \(\sum^{\infty}_{k = 0} {}_{k}E_x\) =
  \(\frac{1 - A_x}{d}\)

  \begin{itemize}
  \tightlist
  \item
    With a constant force of mortality: \(\ddot{a}_x\) =
    \(\frac{1 + i}{q + i}\)
  \end{itemize}
\item
  It is also true that: \(\ddot{a}_x\) =
  \(\sum^{\infty}_{k = 0} (\ddot{a}_{\overline{k+1}|}){}_{k|}q_x\)
\item
  \(Var(Y_d)\) = \(\frac{^2A_x - [A_x]^2}{d^2}\)
\item
  Recursion: \(\ddot{a}_x\) = 1 + \(vp_x\ddot{a}_{x + 1}\)
\end{itemize}

\textbf{Whole Life Annuity-Immediate}:

\begin{itemize}
\item
  \(Y_i\) = \(a_{\overline{K_x}|}\) = \(\frac{1 - v^{K_x}}{i}\)

  So: \(Y_i\) = \(\frac{1 - (1 + i)Z}{i}\), where \(Z\) is the present
  value random variable for an annual whole life insurance of 1 on (x).
\item
  \(E(Y_i)\) = \(a_x\) = \(\sum^{\infty}_{k = 1} {}_{k}E_x\) =
  \(\ddot{a}_x\) - 1
\item
  \(E(Y_i)\) = \(a_x\) = \(\frac{1 - (1 + i)A_x}{i}\)
\item
  \(Var(Y_i)\) = \(\frac{^2A_x - [A_x]^2}{d^2}\) (same as an annual
  whole life annuity-due)
\item
  Recursion: \(a_x\) = \(vp_x(1 + a_{x + 1})\)
\end{itemize}

\subsection{Temporary Life Annuity of 1 per Year on
(x)}\label{temporary-life-annuity-of-1-per-year-on-x}

\textbf{Continuous Temporary Life Annuity}:

\begin{itemize}
\item
  \[Y = \left\{
    \begin{array}{l l}
      \bar{a}_{\overline{T_x}|}   & \quad \text{for $T_x$ $\le$ n}\\
      \bar{a}_{\overline{n}|}        & \quad \text{for $T_x$ $>$ n}\\
    \end{array} \right.\]

  = \(\frac{1 - Z}{\delta}\),

  where \(Z\) is the present value random variable for a continuous
  n-year endowment insurance of 1 on (x).
\item
  \(E(Y)\) = \(\bar{a}_{x:\overline{n}|}\) =
  \(\int^{n}_{0} {}_{t}E_xdt\) =
  \(\frac{1 - \bar{A}_{x:\overline{n}|}}{\delta}\)

  \begin{itemize}
  \tightlist
  \item
    With a constant force of mortality: \(\bar{a}_{x:\overline{n}|}\) =
    \(\frac{1}{\mu + \delta}\)\([1 - \exp[-(\mu + \delta)n]]\)
  \end{itemize}
\item
  \(Var(Y)\) =
  \(\frac{^2\bar{A}_{x:\overline{n}|} - [\bar{A}_{x:\overline{n}|}]^2}{\delta^2}\)
\end{itemize}

\textbf{Annual Temporary Life Annuity}:

\textbf{Temporary Life Annuity-Due}:

\begin{itemize}
\item
  \[Y_d = \left\{
    \begin{array}{l l}
      \ddot{a}_{\overline{K_x + 1}|}   & \quad \text{for $K_x$ = 0, 1, ..., n - 1}\\
      \ddot{a}_{\overline{n}|}             & \quad \text{for $K_x$ = n, n + 1, ...}\\
    \end{array} \right.\]

  = \(\frac{1 - Z}{d}\),

  where \(Z\) is the present value random variable for an annual n-year
  endowment insurance of 1 on (x).
\item
  \(E(Y_d)\) = \(\ddot{a}_{x:\overline{n}|}\) =
  \(\sum^{n - 1}_{k = 0} {}_{k}E_x\) =
  \(\frac{1 - A_{x:\overline{n}|}}{d}\)
\item
  \(Var(Y_d)\) =
  \(\frac{^2 A_{x:\overline{n}|} - [A_{x:\overline{n}|}]^2}{d^2}\)
\item
  Recursion: \(\ddot{a}_{x:\overline{n}|}\) = 1 +
  \(vp_x\ddot{a}_{x + 1:\overline{n-1}|}\)
\end{itemize}

\textbf{Temporary Life Annuity-Immediate}:

\begin{itemize}
\item
  \[Y_i = \left\{
    \begin{array}{l l}
      a_{\overline{K_x}|}   & \quad \text{for $K_x$ = 0, 1, ..., n - 1}\\
      a_{\overline{n}|}             & \quad \text{for $K_x$ = n, n + 1, ...}\\
    \end{array} \right.\]
\item
  \(E(Y_i)\) = \(a_{x :\overline{n}|}\) = \(\sum^{n}_{k = 1} {}_{k}E_x\)
  = \(\ddot{a}_{x:\overline{n}|}\) - 1 + \({}_{n}E_x\)
\item
  Note: \(\ddot{a}_{x:\overline{n}|}\) = 1 +
  \(a_{x :\overline{n - 1}|}\)
\item
  \(Var(Y_i)\) =
  \(\frac{^2 A_{x:\overline{n+1}|} - [A_{x:\overline{n+1}|}]^2}{d^2}\)
\item
  Recursion: \(a_{x :\overline{n}|}\) =
  \(vp_x(1 + a_{x + 1 :\overline{n - 1}|})\)
\end{itemize}

\subsection{Deferred Whole Life Annuity of 1 per Year on
(x)}\label{deferred-whole-life-annuity-of-1-per-year-on-x}

\textbf{Continuous Deferred Whole Life Annuity}:

\begin{itemize}
\item
  \[Y = \left\{
    \begin{array}{l l}
      0               & \quad \text{for $T_x$ $\le$ u}\\
      v^u \bar{a}_{\overline{T_x - u}|}         & \quad \text{for $T_x$ $>$ u}\\
    \end{array} \right.\]
\item
  \(E(Y)\) = \({}_{u|}\bar{a}_x\) = \(\int^{\infty}_u {}_{t}E_x\) =
  \(\bar{a}_x\) - \(\bar{a}_{x:\overline{u}|}\) =
  \({}_{u}E_x\)\(\bar{a}_{x + u}\)
\end{itemize}

\textbf{Annual Deferred Whole Life Annuity}:

\textbf{Deferred Whole Life Annuity-Due}:

\begin{itemize}
\item
  \[Y_d = \left\{
    \begin{array}{l l}
      0                       & \quad \text{for $K_x$ = 0, 1, ..., u - 1}\\
      v^u \ddot{a}_{\overline{K_x + 1 - u}|}             & \quad \text{for $K_x$ = u, u + 1, ...}\\
    \end{array} \right.\]
\item
  \(E(Y_d)\) = \({}_{u|}\ddot{a}_x\) =
  \(\sum^{\infty}_{k = u} {}_{k}E_x\) = \(\ddot{a}_x\) -
  \(\ddot{a}_{x:\overline{u}|}\) = \({}_{u}E_x\)\(\ddot{a}_{x + u}\)
\item
  Recursion: \({}_{u|}\ddot{a}_x\) = 0 +
  \(vp_x({}_{u - 1|}\ddot{a}_{x + 1})\)
\end{itemize}

\textbf{Deferred Whole Life Annuity-Immediate}:

\begin{itemize}
\item
  \[Y_i = \left\{
    \begin{array}{l l}
      0                       & \quad \text{for $K_x$ = 0, 1, ..., u - 1}\\
      v^u a_{\overline{K_x - u}|}             & \quad \text{for $K_x$ = u, u + 1, ...}\\
    \end{array} \right.\]
\item
  \(E(Y_i)\) = \({}_{u|}a_x\) = \(\sum^{\infty}_{k = u + 1} {}_{k}E_x\)
  = \(a_x\) - \(a_{x :\overline{u}|}\) = \({}_{u}E_x\)\(a_{x + u}\)
\item
  Note: \({}_{u|}\ddot{a}_x\) = \({}_{u|}a_x\) + \({}_{u}E_x\)
\item
  Recursion: \({}_{u|}a_x\) = 0 + \(vp_x({}_{u - 1|}a_{x + 1})\)
\end{itemize}

\subsection{Deferred Temporary Life Annuity of 1 per Year on
(x)}\label{deferred-temporary-life-annuity-of-1-per-year-on-x}

\textbf{Continuous Deferred Temporary Life Annuity}:

\begin{itemize}
\item
  \[Y = \left\{
    \begin{array}{l l l}
      0                      & \quad \text{for $T_x$ $\le$ u}\\
      v^u \bar{a}_{\overline{T_x - u}|}                 & \quad \text{for u $<$ $T_x$ $\le$ u + n}\\
      v^u \bar{a}_{\overline{n}|}                       & \quad \text{for $T_x$ $>$ u + n}\\
    \end{array} \right.\]
\item
  \(E(Y)\) = \({}_{u|n}\bar{a}_x\) =
  \({}_{u|}\bar{a}_{x:\overline{n}|}\) = \(\int^{u + n}_u {}_{t}E_x dt\)
  = \(\bar{a}_{x:\overline{u +n}|}\) - \(\bar{a}_{x:\overline{u}|}\) =
  \({}_{u}E_x\)\(\bar{a}_{x + u:\overline{n}|}\)
\end{itemize}

\textbf{Annual Deferred Temporary Life Annuity}:

\textbf{Deferred Temporary Life Annuity-Due}:

\begin{itemize}
\item
  \[Y_d = \left\{
    \begin{array}{l l l}
      0                      & \quad \text{for $K_x$ = 0, 1, ..., u - 1}\\
      v^u \ddot{a}_{\overline{K_x + 1 - u}|}                 & \quad \text{for $K_x$ = u, u + 1, ..., u + n - 1}\\
      v^u \ddot{a}_{\overline{n}|}                       & \quad \text{for $K_x$ = u + n, u + n + 1, ...}\\
    \end{array} \right.\]
\item
  \(E(Y_d)\) = \({}_{u|n}\ddot{a}_x\) =
  \({}_{u|}\ddot{a}_{x:\overline{n}|}\) =
  \(\sum^{u + n - 1}_{k = u} {}_{k}E_x\) =
  \(\ddot{a}_{x:\overline{u +n}|}\) - \(\ddot{a}_{x:\overline{u}|}\) =
  \({}_{u}E_x\)\(\ddot{a}_{x + u:\overline{n}|}\)
\item
  Recursion: \({}_{u|n}\ddot{a}_x\) = 0 +
  \(vp_x({}_{u - 1|n}\ddot{a}_{x + 1})\)
\end{itemize}

\textbf{Deferred Temporary Life Annuity-Immediate}:

\begin{itemize}
\item
  \[Y_i = \left\{
    \begin{array}{l l l}
      0                      & \quad \text{for $K_x$ = 0, 1, ..., u - 1}\\
      v^u a_{\overline{K_x - u}|}                 & \quad \text{for $K_x$ = u, u + 1, ..., u + n - 1}\\
      v^u a_{\overline{n}|}                       & \quad \text{for $K_x$ = u + n, u + n + 1, ...}\\
    \end{array} \right.\]
\item
  \(E(Y_i)\) = \({}_{u|n}a_x\) = \({}_{u|}a_{x :\overline{n}|}\) =
  \(\sum^{u + n}_{k = u + 1} {}_{k}E_x\) = \(a_{x :\overline{u + n}|}\)
  - \(a_{x :\overline{u}|}\) = \({}_{u}E_x\)\(a_{x + u :\overline{n}|}\)
\item
  Recursion: \({}_{u|n}a_x\) = 0 + \(vp_x({}_{u - 1|n}a_{x + 1})\)
\end{itemize}

\subsection{Certain and Life Annuity of 1 per Year on
(x)}\label{certain-and-life-annuity-of-1-per-year-on-x}

\textbf{Continuous Certain and Life Annuity}:

\begin{itemize}
\item
  \[Y = \left\{
    \begin{array}{l l}
      \bar{a}_{\overline{n}|}               & \quad \text{for $T_x$ $\le$ n}\\
      \bar{a}_{\overline{T_x}|}         & \quad \text{for $T_x$ $>$ n}\\
    \end{array} \right.\]
\item
  \(E(Y)\) = \(\bar{a}_{\overline{x: \overline{n}|}}\) =
  \(\bar{a}_{\overline{n}|}\) + \({}_{n|}\bar{a}_x\)
\end{itemize}

\textbf{Annual Certain and Life Annuity}:

\textbf{Certain and Life Annuity-Due}:

\begin{itemize}
\item
  \[Y_d = \left\{
    \begin{array}{l l}
      \ddot{a}_{\overline{n}|}               & \quad \text{for $K_x$ = 0, 1, ..., n - 1}\\
      \ddot{a}_{\overline{K_x + 1}|}         & \quad \text{for $K_x$ = n, n + 1, ...}\\
    \end{array} \right.\]
\item
  \(E(Y_d)\) = \(\ddot{a}_{\overline{x: \overline{n}|}}\) =
  \(\ddot{a}_{\overline{n}|}\) + \({}_{n|}\ddot{a}_x\)
\item
  Recursion: \(\ddot{a}_{\overline{x: \overline{n}|}}\) = 1 +
  \(vq_x\ddot{a}_{\overline{n - 1}|}\) +
  \(vp_x\ddot{a}_{\overline{x + 1: \overline{n - 1}|}}\)
\end{itemize}

\textbf{Certain and Life Annuity-Immediate}:

\begin{itemize}
\item
  \[Y_i = \left\{
    \begin{array}{l l}
      a_{\overline{n}|}               & \quad \text{for $K_x$ = 0, 1, ..., n - 1}\\
      a_{\overline{K_x}|}         & \quad \text{for $K_x$ = n, n + 1, ...}\\
    \end{array} \right.\]
\item
  \(E(Y_i)\) = \(a_{\overline{x: \overline{n}|}}\) =
  \(a_{\overline{n}|}\) + \({}_{n|}a_x\)
\item
  Recursion: \(a_{\overline{x: \overline{n}|}}\) = \(v\) +
  \(vq_xa_{\overline{n - 1}|}\) +
  \(vp_x a_{\overline{x + 1: \overline{n - 1}|}}\)
\end{itemize}

\section{Exercises}\label{exercises-2}

5.1. Assume: \(\mu_x(t)\) = 0.02 for \(t\) \(>\) 0 and \(\delta\) =
0.05.

Calculate: \(\bar{a}_{x:\overline{15}|}\).

(A) 8.70 (B) 8.90 (C) 9.10 (D) 9.30 (E) 9.50

5.2. Assume mortality follows: \(l_x\) = 100(110 - \(x\)) for 0 \(\le\)
\(x\) \(\le\) 110, and \(d\) = 0.05.

Calculate: \({}_{15|}\ddot{a}_{45}\).

(A) 4.40 (B) 4.60 (C) 4.80 (D) 5.00 (E) 5.20

5.3. You are given:

(i) \(\mu\) = \(\delta\) = \(c\), where \(c\) is a positive constant.

(ii) \(^2\bar{a}_x\) = \(\frac{25}{3}\)

Calculate: \(Var(\bar{a}_{\overline{T(x)}|})\).

(A) 50 (B) 52 (C) 54 (D) 56 (E) 58

5.4. You are given:

(i) \({}_{k|}q_{35}\) = 0.005(\(k\) + 1) for \(k\) = 0, 1, 2, 3.

(ii) \(i\) = 0.05

Calculate the actuarial present value of a 4-year temporary life
annuity-due of 100 per year on (35).

(A) 350 (B) 360 (C) 370 (D) 380 (E) 390

5.5. You are given:

(i) \(\delta\) = 0.05

(ii) \[\mu_x = \left\{
  \begin{array}{l l}
    0.05 & \quad \text{for 0 $\le$ x $<$ 50}\\
    0.08 & \quad \text{for x $\ge$ 50}\\
  \end{array} \right.\]

Calculate the actuarial present value of a continuous whole life annuity
of 1 per year on (30).

(A) 9.10 (B) 9.30 (C) 9.50 (D) 9.70 (E) 9.90

5.6. You are given:

\[\delta_t = \left\{
  \begin{array}{l l}
    0.05 & \quad \text{for 0 $\le$ t $<$ 10}\\
    0.07 & \quad \text{for t $\ge$ 10}\\
  \end{array} \right.\]

\[\mu_x(t) = \left\{
  \begin{array}{l l}
    0.02 & \quad \text{for 0 $\le$ t $<$ 10}\\
    0.03 & \quad \text{for t $\ge$ 10}\\
  \end{array} \right.\]

Calculate the expected present value of a continuous 10-year certain and
life annuity on (x) of 1 per year.

(A) 11.60 (B) 11.90 (C) 12.20 (D) 12.50 (E) 12.80

5.7. You are given:

(i) \(\mu_x(t)\) = 0.02 for \(t\) \(>\) 0

(ii) \(\delta\) = 0.06

Calculate: Pr{[}\(\bar{a}_{\overline{T_{x}}|}\) \(>\)
\(\bar{a}_{x}\){]}.

(A) 0.42 (B) 0.55 (C) 0.63 (D) 0.84 (E) 0.91

5.8. You are given:

(i) The force of mortality is constant.

(ii) \(\bar{A}_x\) = 0.428571

(iii) \(\bar{A}_{x:\overline{10}|}^{1}\) = 0.215749

(iv) \(Y\) is the present value random variable for a continuous 10-year
temporary life annuity of 500 per year on (x).

Calculate: \(E(Y)\).

(A) 3550 (B) 3600 (C) 3650 (D) 3700 (E) 3750

5.9. You are given:

(i) Mortality for a standard life aged 40, denoted as \(S\), is such
that:

\(q^S_x\) = 0.032 for \(x\) = 40, 41, 42, \ldots{}

(ii) Mortality for a certain life aged 40, denoted as \(C\), is such
that:

\(q^C_{40}\) = 0.048 and \(q^C_x\) = 0.032 for \(x\) = 41, 42, 43,
\ldots{}

(iii) \(d\) = 0.05

Calculate: \(\ddot{a}^S_{40}\) - \(\ddot{a}^C_{40}\).

(A) 0.16 (B) 0.17 (C) 0.18 (D) 0.19 (E) 0.20

5.10. You are given:

(i) \(\mu_x(t)\) is the force of mortality associated with the
Illustrative Life Table.

(ii) \(i\) = 0.05

Calculate the single benefit premium for a 3-year temporary life
annuity-immediate of 1000 per year on (30) payable annually, assuming
that the force of mortality used is equal to \(\mu_{30}(t)\) + 0.20 for
0 \(\le\) \(t\) \(\le\) 3.

(A) 1860 (B) 1900 (C) 1940 (D) 1980 (E) 2020

5.11. A fund is created such that:

(i) There are 60 lives each age 30.

(ii) Each life receives payments of 100 per year for life, payable
annually, beginning immediately.

(iii) Mortality follows the Illustrative Life Table.

(iv) The lifetimes are independent.

(v) \(i\) = 0.06

(vi) The amount of the fund is determined, using the normal
approximation, such that the probability that the fund is sufficient to
make all payments is 99\%.

Calculate the initial amount of the fund.

(A) 98,000 (B) 98,500 (C) 99,000 (D) 99,500 (E) 100,000

5.12. For a group of individuals all age x:

(i) 35\% are smokers and 65\% are non-smokers.

(ii) The constant force of mortality for smokers is 0.08.

(iii) The constant force of mortality for non-smokers is 0.04.

(iv) \(\delta\) = 0.06

Calculate \(Var[\bar{a}_{\overline{T(x)}|}]\) for an individual chosen
at random from this group.

(A) 25 (B) 26 (C) 27 (D) 28 (E) 29

5.13. Suppose \(Z\) is the present value random variable for a 2-year
pure endowment of 1 on (x). You are given:

(i) \(v\) = 0.95 and \(p_x\) = 0.98

(ii) \(A_x\) = 0.45 and \(\ddot{a}_{x + 2}\) = 10.68

Calculate: \(Var(Z)\).

(A) 0.045 (B) 0.050 (C) 0.055 (D) 0.060 (E) 0.065

5.14. Cody, age 25, and Ted, age 30, have each won the actuarial
lottery:

(i) Cody has decided to collect his winnings via a 20-year temporary
life annuity-due, which pays 400,000 each year.

(ii) Ted has decided to collect his winnings via a 20-year certain and
life annuity-due, which pays \(K\) each year.

(iii) Mortality for both Cody and Ted follows the Illustrative Life
Table, and \(i\) = 0.06.

The expected present values of Cody's annuity and Ted's annuity are both
equal. Calculate: \(K\).

(A) 281,000 (B) 286,000 (C) 291,000 (D) 295,000 (E) 299,000

5.15. Consider a special whole life annuity on (x) which pays \(R\) at
the beginning of the first year, \(2R\) at the beginning of the second
year, and \(3R\) at the beginning of each year thereafter. You are also
given:

(i) The actuarial present value of this annuity is 3333.

(ii) \(i\) = 0.05

(iii) \(p_x\) = 0.98 and \(p_{x + 1}\) = 0.97

(iv) \(\ddot{a}_{x + 2}\) = 31.105

Calculate: \(R\).

(A) 30 (B) 35 (C) 40 (D) 45 (E) 50

5.16. Let \(Y\) denote the present value random variable for a whole
life annuity on (x) of \(R\) per year payable continuously each year:

(i) \(\delta\) = 0.050

(ii) \(\mu_x(t)\) = 0.035 for \(t\) \(\ge\) 0

(iii) The expected value of \(Y\) is 1.64\% of the variance of \(Y\).

Calculate: \(R\).

(A) 12 (B) 14 (C) 16 (D) 18 (E) 20

5.17. You are given the following portfolio of mutually independent
lives:

(i) 50 lives age 65 purchase a whole life annuity-immediate with an
annual payment of 30,000.

(ii) 20 lives age 75 purchase a whole life annuity-immediate with an
annual payment of 20,000.

Mortality follows the Illustrative Life Table, and \(i\) = 0.06.

Let \(S\) be the present value for the total payments on the portfolio.

Calculate the 95th percentile of the distribution of \(S\), in millions,
using the normal approximation.

(A) 16.9 (B) 17.2 (C) 17.5 (D) 17.8 (E) 18.1

5.18. You are given:

(i) \(\mu_x\) = \(\frac{1}{90 - x}\) for 0 \(\le\) \(x\) \(<\) 90

(ii) \(i\) = 0

Calculate: \(\ddot{a}_{30}\).

(A) 29.0 (B) 29.5 (C) 30.0 (D) 30.5 (E) 31.0

5.19. Paul, aged 35, has just taken out a home mortgage loan where he
will pay 12,000 at the end of each year for 25 years.

Paul was also required to purchase a life insurance policy that will pay
any remaining payments should he die within the 25-year period.

Paul has mortality that follows the Illustrative Life Table. The
effective annual interest rate is 6\%.

Calculate the expected present value of the life insurance policy.

(A) 5150 (B) 5250 (C) 5350 (D) 5450 (E) 5550

5.20. You are given:

(i) Mortality follows:

(ii) \(d\) = 0.03

Calculate the probability that the present value of a 5-year temporary
life annuity-due of 500 per year on (30) exceeds its actuarial present
value.

(A) 0.45 (B) 0.50 (C) 0.55 (D) 0.60 (E) 0.65

5.21. You are given:

(i) \(\delta\) = 0.04

(ii) \(\mu_x\) = 0.0003\((1.05)^x\) for \(x\) \(\ge\) 0

Calculate the expected present value of a 2-year deferred 2-year
temporary life annuity-immediate of 100 per year on (34).

(A) 167 (B) 169 (C) 171 (D) 173 (E) 175

5.22. For a 10-year deferred 10-year continuous temporary life annuity
of 1000 per year on (x):

(i) \[\delta_t = \left\{
  \begin{array}{l l}
    0.06 & \quad \text{for t $\le$ 6}\\
    0.07 & \quad \text{for t $>$ 6}\\
  \end{array} \right.\]

(ii) \[\mu_x(t) = \left\{
  \begin{array}{l l}
    0.025 & \quad \text{for t $\le$ 6}\\
    0.035 & \quad \text{for t $>$ 6}\\
  \end{array} \right.\]

Calculate the single benefit premium for this annuity.

(A) 2320 (B) 2360 (C) 2400 (D) 2440 (E) 2480

5.23. Consider a policy on (40) that provides the following benefits:

(i) A whole life annuity-due of 500 per year payable annually.

(ii) A death benefit of 5000 payable at the end of the year of death.

Furthermore:

(iii) \(i\) = 0.06

(iv) Mortality follows the Illustrative Life Table.

Calculate the standard deviation of the present value random variable
for this policy.

(A) 515 (B) 530 (C) 545 (D) 560 (E) 575

5.24. You are given:

\begin{enumerate}
\def\labelenumi{(\roman{enumi})}
\item
\end{enumerate}

(ii) \(i\) = 0.05

Calculate the actuarial accumulated value at the end of the fifth year
of a 5-year temporary life annuity-immediate of 100 per year payable
annually on (70).

(A) 310 (B) 425 (C) 540 (D) 665 (E) 785

5.25. Consider a whole life annuity-due of 1 per year payable annually
on (x):

(i) \(v\) = 0.965

(ii) \({}_{10}p_x\) = 0.920

(iii) \(\ddot{a}_{x + 11}\) = 11.36

Suppose \(q_{x + 10}\) is increased to \(q_{x + 10}\) + 0.100.

Calculate the change in the expected present value of the annuity.

(A) - 0.71 (B) - 0.41 (C) - 0.26 (D) 0.26 (E) 0.58

5.26. Consider the following special annuity on (x) payable annually:

(i) The payment at time \(k\), \(\pi_k\), is such that:

\[\pi_k = \left\{
  \begin{array}{l l}
    1000vq_{x + k} & \quad \text{for $k$ = 0, 1, ..., 19}\\
    0             & \quad \text{for $k$ = 20, 21, ...}\\
  \end{array} \right.\]

(ii) \(i\) = 0.045

(iii) \({}_{20}p_x\) = 0.945

(iv) \(a_x\) = 18.23

(v) \(a_{x + 20}\) = 13.94

Calculate the expected present value of this annuity.

(A) 30 (B) 32 (C) 34 (D) 36 (E) 38

5.27. Consider a policy on (x) that provides the following benefits:

(i) 1000 per year payable continuously each year while (x) is alive.

(ii) \(S\) payable at the moment of death of (x).

Furthermore:

(iii) \(\delta\) = 0.05

(iv) \(\mu_x(t)\) = 0.03 for \(t\) \(>\) 0.

(v) \(W\) denotes the present value random variable for this policy.

Determine the value of \(S\) that minimizes the variance of \(W\).

(A) 5000 (B) 10,000 (C) 15,000 (D) 20,000 (E) 25,000

\subsection{Answers to Exercises}\label{answers-to-exercises-2}

5.1. D 5.26. B

5.2. B 5.27. D

5.3. B

5.4. C

5.5. D

5.6. E

5.7. C

5.8. B

5.9. D

5.10. A

5.11. C

5.12. A

5.13. A

5.14. E

5.15. C

5.16. E

5.17. B

5.18. D

5.19. C

5.20. B

5.21. D

5.22. D

5.23. E

5.24. E

5.25. A

\section{Past Exam Questions}\label{past-exam-questions-2}

\begin{itemize}
\item
  Exam MLC, Spring 2015: \#7
\item
  Exam MLC, Fall 2014: \#5
\item
  Exam MLC, Spring 2014: \#6
\item
  Exam MLC, Fall 2013: \#1, 5
\item
  Exam 3L, Fall 2013: \#12, 13
\item
  Exam MLC, Spring 2013: \#21
\item
  Exam 3L, Spring 2013: \#12
\item
  Exam 3L, Fall 2012: \#12
\item
  Exam MLC, Spring 2012: \#15
\item
  Exam MLC, Sample Questions: \#11, 25, 35, 45, 55, 63, 67, 79, 86, 88,
  113, 114, 126, 130, 140, 146, 166, 186, 192, 196, 209, 210, 229, 285
\item
  Exam 3L, Fall 2011: \#12
\item
  Exam 3L, Spring 2011: \#12
\item
  Exam 3L, Spring 2010: \#15
\item
  Exam 3L, Fall 2009: \#12
\item
  Exam 3L, Spring 2009: \#12, 13
\item
  Exam 3L, Fall 2008: \#20, 21
\item
  Exam MLC, Spring 2007: \#2, 17, 24, 29
\end{itemize}

\chapter{Annuities II}\label{annuities-ii}

\section{m-thly Annuities-Due}\label{m-thly-annuities-due}

Here, we consider a \textbf{discrete} life annuity where each payment is
provided at the beginning of the \(m\)-th of a year, conditional on
survival. The value of \(m\) is typically equal to 2 (half-year), 4
(quarter of a year), or 12 (month).

Note:

\begin{itemize}
\item
  An \(m\)-thly life annuity-due is such that each payment is made at
  the beginning of each \(m\)-th of a year (for a total of
  \(m (K^{(m)}_x + \frac{1}{m})\) payments, in years).
\item
  The general exact formula for the expected present value of an
  \(m\)-thly life annuity-due on (x) that pays \(\pi_{\frac{k}{m}}\) at
  time \(\frac{k}{m}\) years (\(k\) = 0, 1, 2, \ldots{}) is:

  \(E(Y_d)\) =
  \(\sum^{\infty}_{k = 0} \pi_{\frac{k}{m}} v^{\frac{k}{m}}{}_{\frac{k}{m}}p_x\)
  = \(\sum^{\infty}_{k = 0} \pi_{\frac{k}{m}} ({}_{\frac{k}{m}}E_x)\).
  Furthermore:

  ``Say the payment is \(\pi_{\frac{k}{m}}\) at time \(\frac{k}{m}\).
  Then the present value of this benefit is
  \(\pi_{\frac{k}{m}} v^{\frac{k}{m}}\), and the expected present value
  of this benefit is
  \(\pi_{\frac{k}{m}} v^{\frac{k}{m}}{}_{\frac{k}{m}}p_x\) ((x) has to
  survive to time \(\frac{k}{m}\) in order for \(\pi_{\frac{k}{m}}\) to
  be made at that time). Summing over all possible payment times
  provides the overall expected present value.''
\item
  The International Actuarial Notation for \(E(Y_d)\) often contains an
  \(\ddot{a}^{(m)}\), which indicates that the expected present value is
  for an m-thly life annuity-due. For example, \(\ddot{a}^{(m)}_x\)
  denotes the expected present value of a whole life annuity-due of 1
  per year on (x), payable in equal installments of \(\frac{1}{m}\) at
  the beginning of each \(m\)-th of the year.
\item
  \(^2\ddot{a}^{(m)}\) is \(\ddot{a}^{(m)}\) evaluated at double the
  force of interest.
\item
  Often, we do not use the exact formulas to calculate expected present
  values for an \(m\)-thly life annuity-due. Rather, we approximate
  these expected present values from the corresponding annual life
  annuity-due expected present values using one of two assumptions:

  \begin{itemize}
  \item
    \textbf{UDD}: deaths are uniformly distributed within each year of
    age. In UDD formulas that approximate \(m\)-thly life annuities-due,
    we use the following functions:

    \(\alpha(m)\) = \(\frac{id}{i^{(m)}d^{(m)}}\) and \(\beta(m)\) =
    \(\frac{i - i^{(m)}}{i^{(m)}d^{(m)}}\)

    Let \(m\) approach infinity. In UDD formulas that approximate
    continuous life annuities, we use the following functions:

    \(\alpha(\infty)\) = \(\frac{id}{\delta^2}\) and \(\beta(\infty)\) =
    \(\frac{i - \delta}{\delta^2}\).

    You are provided a table of \(\alpha(m)\) and \(\beta(m)\) for
    various values of \(m\) at \(i\) = 0.06 during Exam MLC. Please
    refer to the web link to Exam MLC tables provided in Appendix A of
    this study supplement.
  \item
    \textbf{Woolhouse's Formula}: based on series expansions. For
    example, Woolhouse's Formula with three terms for an \(m\)-thly
    whole life annuity-due of 1 per year on (x) is:

    \(\ddot{a}^{(m)}_x\) = \(\ddot{a}_x\) - \(\frac{m - 1}{2m}\) -
    \(\frac{m^2 - 1}{12m^2}\)\((\delta + \mu_x)\), and

    Woolhouse's Formula with two terms for an \(m\)-thly whole life
    annuity-due of 1 per year on (x) is:

    \(\ddot{a}^{(m)}_x\) = \(\ddot{a}_x\) - \(\frac{m - 1}{2m}\) (this
    approximates the UDD formula).

    Furthermore, \(\mu_x\) can be approximated as
    -\(\frac{1}{2}\)\((\ln p_{x - 1} + \ln p_x)\).
  \end{itemize}
\item
  Note: The formulas discussed below reduce to the corresponding annual
  life annuity-due formulas in \textbf{Annuities I} when \(m\) = 1.
\end{itemize}

\begin{itemize}
\item
  Note: While unlikely, it is possible that \(m\)-thly life
  annuities-immediate could be tested; each payment is made at the end
  of each \(m\)-th of a year (for a total of \(m (K^{(m)}_x)\) payments,
  in years).

  You can use the following relations:

  \begin{itemize}
  \item
    \(a^{(m)}_x\) = \(\ddot{a}^{(m)}_x\) - \(\frac{1}{m}\)
  \item
    \(a^{(m)}_{x :\overline{n}|}\) =
    \(\ddot{a}^{(m)}_{x :\overline{n}|}\) - \(\frac{1}{m}\)(1 -
    \({}_{n}E_x)\)
  \item
    \({}_{n|}a^{(m)}_x\) = \(a^{(m)}_x\) -
    \(a^{(m)}_{x :\overline{n}|}\)
  \item
    \(a^{(m)}_{\overline{x :\overline{n}|}}\) =
    \(a^{(m)}_{\overline{n}|}\) + \({}_{n|}a^{(m)}_x\)
  \end{itemize}
\end{itemize}

\subsection{m-thly Whole Life Annuity-Due of 1 per Year on
(x)}\label{m-thly-whole-life-annuity-due-of-1-per-year-on-x}

\begin{itemize}
\item
  \(Y_d\) = \(\ddot{a}^{(m)}_{\overline{K^{(m)}_x + \frac{1}{m}}| }\) =
  \(\frac{1 - v^{K^{(m)}_x + \frac{1}{m}}}{d^{(m)}}\)

  So: \(Y_d\) = \(\frac{1 - Z}{d^{(m)}}\), where \(Z\) is the present
  value random variable for an m-thly whole life insurance of 1 on (x).
\item
  \(E(Y_d)\) = \(\ddot{a}^{(m)}_x\) =
  \(\sum^{\infty}_{k = 0} \frac{1}{m}{}_{\frac{k}{m}}E_x\) =
  \(\frac{1 - A^{(m)}_x}{d^{(m)}}\)
\item
  \(Var(Y_d)\) = \(\frac{^2A^{(m)}_x - [A^{(m)}_x]^2}{[d^{(m)}]^2}\)
\item
  Recursion: \(\ddot{a}^{(m)}_x\) = \(\frac{1}{m}\) +
  \(v^{\frac{1}{m}}{}_{\frac{1}{m}}p_x\ddot{a}^{(m)}_{x + \frac{1}{m}}\)
\item
  UDD: \(\ddot{a}^{(m)}_x\) = \(\alpha(m)\)\(\ddot{a}_x\) -
  \(\beta(m)\); \(\bar{a}_x\) = \(\alpha(\infty)\)\(\ddot{a}_x\) -
  \(\beta(\infty)\)
\item
  Woolhouse's Formula with 3 terms: \(\ddot{a}^{(m)}_x\) =
  \(\ddot{a}_x\) - \(\frac{m - 1}{2m}\) -
  \(\frac{m^2 - 1}{12m^2}\)\((\delta + \mu_x)\);

  \(\bar{a}_x\) = \(\ddot{a}_x\) - \(\frac{1}{2}\) -
  \(\frac{1}{12}\)\((\delta + \mu_x)\)
\end{itemize}

\subsection{m-thly Temporary Life Annuity-Due of 1 per Year on
(x)}\label{m-thly-temporary-life-annuity-due-of-1-per-year-on-x}

\begin{itemize}
\item
  \[Y_d = \left\{
    \begin{array}{l l}
      \ddot{a}^{(m)}_{\overline{K^{(m)}_x + \frac{1}{m}}| }   & \quad \text{for $K^{(m)}_x$ = 0, $\frac{1}{m}$, ..., n - $\frac{1}{m}$}\\
      \ddot{a}^{(m)}_{\overline{n}|}             & \quad \text{for $K^{(m)}_x$ = n, n + $\frac{1}{m}$, ...}\\
    \end{array} \right.\] = \(\frac{1 - Z}{d^{(m)}}\)

  where \(Z\) is the present value random variable for an m-thly n-year
  endowment insurance of 1 on (x).
\item
  \(E(Y_d)\) = \(\ddot{a}^{(m)}_{x :\overline{n}|}\) =
  \(\sum^{mn - 1}_{k = 0} \frac{1}{m} {}_{\frac{k}{m}}E_x\) =
  \(\frac{1 - A^{(m)}_{x :\overline{n}|}}{d^{(m)}}\)
\item
  \(Var(Y_d)\) =
  \(\frac{^2A^{(m)}_{x :\overline{n}|} - [A^{(m)}_{x :\overline{n}|}]^2}{[d^{(m)}]^2}\)
\item
  Recursion: \(\ddot{a}^{(m)}_{x :\overline{n}|}\) = \(\frac{1}{m}\) +
  \(v^{\frac{1}{m}}{}_{\frac{1}{m}}p_x\ddot{a}^{(m)}_{x + \frac{1}{m} :\overline{n - \frac{1}{m}}| }\)
\item
  UDD: \(\ddot{a}^{(m)}_{x :\overline{n}|}\) =
  \(\alpha(m)\)\(\ddot{a}_{x:\overline{n}|}\) -
  \(\beta(m)\)\((1 - {}_{n}E_x)\);

  \(\bar{a}_{x:\overline{n}|}\) =
  \(\alpha(\infty)\)\(\ddot{a}_{x:\overline{n}|}\) -
  \(\beta(\infty)\)\((1 - {}_{n}E_x)\)
\item
  Woolhouse's Formula with 3 terms:

  \(\ddot{a}^{(m)}_{x :\overline{n}|}\) = \(\ddot{a}_{x:\overline{n}|}\)
  - \(\frac{m - 1}{2m}\)\((1 - {}_{n}E_x)\) -
  \(\frac{m^2 - 1}{12m^2}\)\((\delta + \mu_x - {}_{n}E_x(\delta + \mu_{x + n}))\);

  \(\bar{a}_{x:\overline{n}|}\) = \(\ddot{a}_{x:\overline{n}|}\) -
  \(\frac{1}{2}\)\((1 - {}_{n}E_x)\) -
  \(\frac{1}{12}\)\((\delta + \mu_x - {}_{n}E_x(\delta + \mu_{x + n}))\)

  I would not memorize these UDD and Woolhouse's formulas. Just know the
  \(m\)-thly whole life results and use:

  \(\ddot{a}^{(m)}_{x :\overline{n}|}\) = \(\ddot{a}^{(m)}_{x}\) -
  \({}_{n}E_x\)\(\ddot{a}^{(m)}_{x + n}\).
\end{itemize}

\subsection{m-thly Deferred Whole Life Annuity-Due of 1 per Year on
(x)}\label{m-thly-deferred-whole-life-annuity-due-of-1-per-year-on-x}

\begin{itemize}
\item
  \[Y_d = \left\{
    \begin{array}{l l}
      0                       & \quad \text{for $K^{(m)}_x$ = 0, $\frac{1}{m}$, ..., u - $\frac{1}{m}$}\\
      v^u \ddot{a}^{(m)}_{\overline{K^{(m)}_x + \frac{1}{m} - u}| }             & \quad \text{for $K^{(m)}_x$ = u, u + $\frac{1}{m}$, ...}\\
    \end{array} \right.\]
\item
  \(E(Y_d)\) = \({}_{u|}\ddot{a}^{(m)}_x\) =
  \(\sum^{\infty}_{k = mu} \frac{1}{m} {}_{\frac{k}{m}}E_x\) =
  \(\ddot{a}^{(m)}_x\) - \(\ddot{a}^{(m)}_{x :\overline{u}|}\) =
  \({}_{u}E_x\)\(\ddot{a}^{(m)}_{x + u}\)
\item
  Recursion: \({}_{u|}\ddot{a}^{(m)}_x\) = 0 +
  \(v^{\frac{1}{m}}{}_{\frac{1}{m}}p_x({}_{u - \frac{1}{m}|}\ddot{a}^{(m)}_{x + \frac{1}{m}})\)
\item
  UDD: \({}_{u|}\ddot{a}^{(m)}_x\) = \(\alpha(m)\)\({}_{u|}\ddot{a}_x\)
  - \(\beta(m)\)\({}_{u}E_x\); \({}_{u|}\bar{a}_x\) =
  \(\alpha(\infty)\)\({}_{u|}\ddot{a}_x\) -
  \(\beta(\infty)\)\({}_{u}E_x\)
\item
  Woolhouse's Formula with 3 terms: \({}_{u|}\ddot{a}^{(m)}_x\) =
  \({}_{u|}\ddot{a}_x\) - \(\frac{m - 1}{2m}\)\({}_{u}E_x\) -
  \(\frac{m^2 - 1}{12m^2}\)\({}_{u}E_x(\delta + \mu_{x + u})\);

  \({}_{u|}\bar{a}_x\) = \({}_{u|}\ddot{a}_x\) -
  \(\frac{1}{2}\)\({}_{u}E_x\) -
  \(\frac{1}{12}\)\({}_{u}E_x(\delta + \mu_{x + u})\)

  I would not memorize these UDD and Woolhouse's formulas. Just know the
  \(m\)-thly whole life results and use: \({}_{n|}\ddot{a}^{(m)}_{x}\) =
  \({}_{n}E_x\)\(\ddot{a}^{(m)}_{x + n}\).
\end{itemize}

\subsection{m-thly Deferred Temporary Life Annuity-Due of 1 per Year on
(x)}\label{m-thly-deferred-temporary-life-annuity-due-of-1-per-year-on-x}

\begin{itemize}
\item
  \[Y_d = \left\{
    \begin{array}{l l l}
      0                      & \quad \text{for $K^{(m)}_x$ = 0, $\frac{1}{m}$, ..., u - $\frac{1}{m}$}\\
      v^u \ddot{a}^{(m)}_{\overline{K^{(m)}_x + \frac{1}{m} - u}| }                 & \quad \text{for $K^{(m)}_x$ = u, u + $\frac{1}{m}$, ..., u + n - $\frac{1}{m}$}\\
      v^u \ddot{a}^{(m)}_{\overline{n}|}                       & \quad \text{for $K^{(m)}_x$ = u + n, u + n + $\frac{1}{m}$, ...}\\
    \end{array} \right.\]
\item
  \(E(Y_d)\) = \({}_{u|n}\ddot{a}^{(m)}_x\) =
  \(\sum^{m(u + n) - 1}_{k = mu} \frac{1}{m} {}_{\frac{k}{m}}E_x\) =
  \(\ddot{a}^{(m)}_{x :\overline{u + n}|}\) -
  \(\ddot{a}^{(m)}_{x :\overline{u}|}\) =
  \({}_{u}E_x\)\(\ddot{a}^{(m)}_{x + u:\overline{n}|}\)
\item
  Recursion: \({}_{u|n}\ddot{a}^{(m)}_x\) = 0 +
  \(v^{\frac{1}{m}}{}_{\frac{1}{m}}p_x({}_{u - \frac{1}{m}|n}\ddot{a}^{(m)}_{x + \frac{1}{m}})\)
\end{itemize}

\subsection{m-thly Certain and Life Annuity-Due of 1 per Year on
(x)}\label{m-thly-certain-and-life-annuity-due-of-1-per-year-on-x}

\begin{itemize}
\item
  \[Y_d = \left\{
    \begin{array}{l l}
      \ddot{a}^{(m)}_{\overline{n}|}               & \quad \text{for $K^{(m)}_x$ = 0, $\frac{1}{m}$, ..., n - $\frac{1}{m}$}\\
      \ddot{a}^{(m)}_{\overline{K^{(m)}_x + \frac{1}{m}}| }         & \quad \text{for $K^{(m)}_x$ = n, n + $\frac{1}{m}$, ...}\\
    \end{array} \right.\]
\item
  \(E(Y_d)\) = \(\ddot{a}^{(m)}_{\overline{x: \overline{n}|}}\) =
  \(\ddot{a}^{(m)}_{\overline{n}|}\) + \({}_{n|}\ddot{a}^{(m)}_x\)
\item
  Recursion: \(\ddot{a}^{(m)}_{\overline{x: \overline{n}|}}\) =
  \(\frac{1}{m}\) +
  \(v^{\frac{1}{m}}{}_{\frac{1}{m}}q_x\ddot{a}^{(m)}_{\overline{n - \frac{1}{m}}| }\)

  \begin{itemize}
  \tightlist
  \item
    \(v^{\frac{1}{m}}{}_{\frac{1}{m}}p_x\ddot{a}^{(m)}_{\overline{x + \frac{1}{m}: \overline{n - \frac{1}{m}}| }}\)
  \end{itemize}
\end{itemize}

\section{Varying Annuities}\label{varying-annuities}

\begin{itemize}
\item
  \textbf{Annually Increasing Annuity:} A life annuity where the annual
  payment is increased arithmetically for each year the annuitant
  survives. For example, an annually increasing temporary life
  annuity-due pays \(R\) at the beginning of the first year, \(2R\) at
  the beginning of the second year given survival of the
  annuitant,\ldots{}, \(nR\) at the beginning of year \(n\) given
  survival of the annuitant.
\item
  \textbf{Annually Decreasing Annuity:} A life annuity where the annual
  payment is decreased arithmetically for each year the annuitant
  survives. For example, an annually decreasing temporary life
  annuity-due pays \(nR\) at the beginning of the first year,
  \((n - 1)R\) at the beginning of the second year given survival of the
  annuitant,\ldots{}, \(R\) at the beginning of year \(n\) given
  survival of the annuitant.
\item
  \textbf{Geometrically Increasing Annuity:} A life annuity where the
  annual payment is increased at a compound rate of \(j\) per year for
  each year the annuitant survives. For example, a geometrically
  increasing whole life annuity-due pays \(R\) at the beginning of the
  first year, \(R(1 +j)\) at the beginning of the second year given
  survival of the annuitant, \(R(1 +j)^2\) at the beginning of the third
  year given survival of the annuitant, etc.
\end{itemize}

\subsection{Annually Increasing Whole Life Annuity on
(x)}\label{annually-increasing-whole-life-annuity-on-x}

\textbf{Continuous Annually Increasing Whole Life Annuity}:

\begin{itemize}
\item
  Provides 1 continuously during the first year, 2 continuously during
  the second year, 3 continuously during the third year, etc.
\item
  \(E(Y)\) = \((I\bar{a})_{x}\) =
  \(\sum^{\infty}_{k = 0} (k + 1){}_{k|}\bar{a}_{x:\overline{1}|}\)
\end{itemize}

\textbf{Continuously Increasing Whole Life Annuity}:

\begin{itemize}
\item
  The payment rate at time \(t\) is \(t\).
\item
  \(E(Y)\) = \((\bar{I}\bar{a})_{x}\) =
  \(\int^{\infty}_0 t{}_{t}E_x dt\)

  \begin{itemize}
  \tightlist
  \item
    With a constant force of mortality: \((\bar{I}\bar{a})_{x}\) =
    \(\frac{1}{(\mu + \delta)^2}\).
  \end{itemize}
\end{itemize}

\textbf{Annually Increasing Whole Life Annuity-Due}:

\begin{itemize}
\item
  Provides 1 at the beginning of the first year, 2 at the beginning of
  the second year, 3 at the beginning of the third year, etc.
\item
  \(E(Y_d)\) = \((I\ddot{a})_{x}\) =
  \(\sum^{\infty}_{k = 0} (k + 1){}_{k}E_x\)

  \begin{itemize}
  \tightlist
  \item
    With a constant force of mortality: \((I\ddot{a})_{x}\) =
    \(\frac{(1 + i)^2}{(q + i)^2}\)
  \end{itemize}
\item
  Recursion: \((I\ddot{a})_x\) = \(\ddot{a}_x\) +
  \(vp_x(I\ddot{a})_{x + 1}\)
\end{itemize}

\textbf{Annually Increasing Whole Life Annuity-Immediate}:

\begin{itemize}
\item
  Provides 1 at the end of the first year, 2 at the end of the second
  year, 3 at the end of the third year, etc.
\item
  \(E(Y_i)\) = \((Ia)_{x}\) = \(\sum^{\infty}_{k = 1} k{}_{k}E_x\)
\item
  Recursion: \((Ia)_x\) = \(a_x\) + \(vp_x(Ia)_{x + 1}\)
\end{itemize}

\subsection{Annually Increasing Temporary Life Annuity on
(x)}\label{annually-increasing-temporary-life-annuity-on-x}

\textbf{Continuous Annually Increasing Temporary Life Annuity}:

\begin{itemize}
\item
  Provides 1 continuously during the first year, 2 continuously during
  the second year, \ldots{}, n continuously during year n.
\item
  \(E(Y)\) = \((I\bar{a})_{x: \overline{n}|}\) =
  \(\sum^{n - 1}_{k = 0} (k + 1){}_{k|}\bar{a}_{x:\overline{1}|}\)
\end{itemize}

\textbf{Continuously Increasing Temporary Life Annuity}:

\begin{itemize}
\item
  The payment rate at time \(t\) is \(t\) during the first n years.
\item
  \(E(Y)\) = \((\bar{I}\bar{a})_{x: \overline{n}|}\) =
  \(\int^{n}_0 t{}_{t}E_x dt\)
\end{itemize}

\textbf{Annually Increasing Temporary Life Annuity-Due}:

\begin{itemize}
\item
  Provides 1 at the beginning of the first year, 2 at the beginning of
  the second year, \ldots{}, n at the beginning of year n.
\item
  \(E(Y_d)\) = \((I\ddot{a})_{x: \overline{n}|}\) =
  \(\sum^{n - 1}_{k = 0} (k + 1){}_{k}E_x\)
\item
  Recursion: \((I\ddot{a})_{x: \overline{n}|}\) =
  \(\ddot{a}_{x: \overline{n}|}\) +
  \(vp_x(I\ddot{a})_{x + 1: \overline{n - 1}|}\)
\end{itemize}

\textbf{Annually Increasing Temporary Life Annuity-Immediate}:

\begin{itemize}
\item
  Provides 1 at the end of the first year, 2 at the end of the second
  year, \ldots{}, n at the end of year n.
\item
  \(E(Y_i)\) = \((Ia)_{x: \overline{n}|}\) =
  \(\sum^{n}_{k = 1} k{}_{k}E_x\)
\item
  Recursion: \((Ia)_{x: \overline{n}|}\) = \(a_{x: \overline{n}|}\) +
  \(vp_x(Ia)_{x + 1: \overline{n - 1}|}\)
\end{itemize}

\subsection{Annually Decreasing Temporary Life Annuity on
(x)}\label{annually-decreasing-temporary-life-annuity-on-x}

\textbf{Continuous Annually Decreasing Temporary Life Annuity}:

\begin{itemize}
\item
  Provides n continuously during the first year, n - 1 continuously
  during the second year, \ldots{}, 1 continuously during year n.
\item
  \(E(Y)\) = \((D\bar{a})_{x: \overline{n}|}\) =
  \(\sum^{n - 1}_{k = 0} (n - k){}_{k|}\bar{a}_{x:\overline{1}|}\)

  Also: \((I\bar{a})_{x: \overline{n}|}\) +
  \((D\bar{a})_{x: \overline{n}|}\) =
  \((n + 1)\bar{a}_{x:\overline{n}|}\).
\end{itemize}

\textbf{Continuously Decreasing Temporary Life Annuity}:

\begin{itemize}
\item
  The payment rate at time \(t\) is \(n\) - \(t\) during the first n
  years.
\item
  \(E(Y)\) = \((\bar{D}\bar{a})_{x: \overline{n}|}\) =
  \(\int^{n}_0 (n - t){}_{t}E_x dt\)

  Also: \((\bar{I}\bar{a})_{x: \overline{n}|}\) +
  \((\bar{D}\bar{a})_{x: \overline{n}|}\) =
  \(n\bar{a}_{x:\overline{n}|}\).
\end{itemize}

\textbf{Annually Decreasing Temporary Life Annuity-Due}:

\begin{itemize}
\item
  Provides n at the beginning of the first year, n - 1 at the beginning
  of the second year, \ldots{}, 1 at the beginning of year n.
\item
  \(E(Y_d)\) = \((D\ddot{a})_{x: \overline{n}|}\) =
  \(\sum^{n - 1}_{k = 0} (n - k){}_{k}E_x\)

  Also: \((I\ddot{a})_{x: \overline{n}|}\) +
  \((D\ddot{a})_{x: \overline{n}|}\) =
  \((n + 1)\ddot{a}_{x:\overline{n}|}\).
\item
  Recursion: \((D\ddot{a})_{x: \overline{n}|}\) = \(n\) +
  \(vp_x(D\ddot{a})_{x + 1: \overline{n - 1}|}\)
\end{itemize}

\textbf{Annually Decreasing Temporary Life Annuity-Immediate}:

\begin{itemize}
\item
  Provides n at the end of the first year, n - 1 at the end of the
  second year, \ldots{}, 1 at the end of year n.
\item
  \(E(Y_i)\) = \((Da)_{x: \overline{n}|}\) =
  \(\sum^{n}_{k = 1} (n - k + 1){}_{k}E_x\)

  Also: \((Ia)_{x: \overline{n}|}\) + \((Da)_{x: \overline{n}|}\) =
  \((n + 1)a_{x: \overline{n}|}\).
\item
  Recursion: \((Da)_{x: \overline{n}|}\) = \(nvp_x\) +
  \(vp_x(Da)_{x + 1: \overline{n - 1}|}\)
\end{itemize}

\subsection{Geometrically Increasing Life Annuity on
(x)}\label{geometrically-increasing-life-annuity-on-x}

\begin{itemize}
\item
  Consider an annual whole life annuity-due where the payment at time 0
  is 1, the payment at time 1 is \((1 + j)\), the payment at time 2 is
  \((1 + j)^2\), etc. There is an effective annual interest rate of
  \(i\).

  Define the modified interest rate: \(i_{\pi}\) =
  \(\frac{1 + i}{1 + j}\) - 1. Then:

  The expected present value of the above annual whole life annuity-due
  is \(\ddot{a}_x\) at \(i_{\pi}\).
\item
  Consider an annual n-year temporary life annuity-due where the payment
  at time 0 is 1, the payment at time 1 is \((1 + j)\), the payment at
  time 2 is \((1 + j)^2\), \ldots{}, the payment at time n - 1 is
  \((1 + j)^{n - 1}\). There is an effective annual interest rate of
  \(i\).

  Define the modified interest rate: \(i_{\pi}\) =
  \(\frac{1 + i}{1 + j}\) - 1. Then:

  The expected present value of the above annual n-year temporary life
  annuity-due is \(\ddot{a}_{x:\overline{n}|}\) at \(i_{\pi}\).
\end{itemize}

\section{Exercises}\label{exercises-3}

6.1. You are given:

(i) Mortality follows the Illustrative Life Table.

(ii) Deaths are uniformly distributed over each year of age.

(iii) \(i\) = 0.06

Calculate: \(\ddot{a}_{35:\overline{20}|}^{(12)}\).

(A) 11.0 (B) 11.2 (C) 11.5 (D) 11.8 (E) 12.0

6.2. You are given:

(i) Mortality follows a select-and-ultimate table, 3-year select period.

(ii) \(\ddot{a}_{[40] + 1}\) = 19.2297

(iii) \(i\) = 0.045

(iv) \(p_{[40]}\) = 0.9987

(v) \(\mu_{[40] + 1}\) = 0.001321

Using Woolhouse's formula with three terms, calculate:
\({}_{1|}\ddot{a}^{(4)}_{[40]}\).

(A) 18.0 (B) 18.1 (C) 18.2 (D) 18.3 (E) 18.4

6.3. Consider a special 20-year temporary life annuity-due on (30) with
annual payments:

(i) The payment for the beginning of year (\(k\) + 1) is: \(\pi_k\) =
\((1.04)^k\) for \(k\) = 0, 1, 2, \ldots{}, 19.

(ii) \(i\) = 0.06

(iii) \(l_x\) = 100 - \(x\) for 0 \(\le\) \(x\) \(\le\) 100

Calculate the single benefit premium.

(A) 14.6 (B) 14.7 (C) 14.8 (D) 14.9 (E) 15.0

6.4. You are given:

(i) \(l_{62}\) = 8,982,404, \(l_{63}\) = 8,915,575, and \(l_{64}\) =
8,842,735

(ii) \(v\) = 0.9569

(iii) \((I\ddot{a})_{62}\) = 158.94

(iv) \((I\ddot{a})_{64}\) = 145.55

Calculate: \(\ddot{a}_{63}\).

(A) 13.2 (B) 13.4 (C) 13.6 (D) 13.8 (E) 14.0

6.5. A fund is established to provide annuity benefits to 500
independent lives all age 35.

You are given:

(i) On January 1, 2012, each life is issued a single premium whole life
annuity. The total payment for each year is 12,000, which is payable in
equal monthly installments in advance.

(ii) Each life has mortality that follows the Illustrative Life Table.

(iii) \(i\) = 0.06

(iv) Deaths are uniformly distributed within each year of age.

Calculate the amount needed in the fund on January 1, 2012, in millions,
so that the probability, as determined by the normal approximation, is
0.99 that the fund will be sufficient to provide these benefits.

(A) 90.0 (B) 90.5 (C) 91.0 (D) 91.5 (E) 92.0

6.6. Consider a special life annuity issued to Jenn, aged 37:

(i) There is a deferral period of 10 years. If Jenn dies during the
deferral period, 80\% of the net single premium is refunded without
interest at the end of the year of death.

(ii) During the 15-year period starting at the end of the deferral
period, 1000 is payable at the beginning of each month while Jenn is
alive. If Jenn is still alive 25 years after issue, 3000 is payable at
the beginning of each month for life.

(iii) Mortality follows the Illustrative Life Table.

(iv) Deaths are uniformly distributed over each year of age.

(v) \(i\) = 0.06

Calculate the net single premium.

(A) 135,500 (B) 136,100 (C) 136,700 (D) 137,300 (E) 137,900

6.7. Consider a 15-year certain and life annuity-due of 24,000 per year
on (65) payable monthly (actual payments are 2000 per month):

(i) Mortality follows the Illustrative Life Table.

(ii) \(i\) = 0.06

(iii) Deaths are uniformly distributed over each year of age.

Calculate the expected present value of this annuity.

(A) 267,900 (B) 268,400 (C) 268,900 (D) 269,400 (E) 269,900

6.8. You are given:

(i) \(Y_1\) is the present value random variable for a 10-year temporary
life annuity-due of 1 per year on a select life aged 40 payable
quarterly.

(ii) \(Y_2\) is the present value random variable for a 10-year certain
and life annuity-due of 1 per year on a select life aged 40 payable
quarterly.

(iii) \(i\) = 0.05

(iv) Mortality follows the Standard Select Survival Model.

(v) Woolhouse's formula with three terms is used to approximate
quarterly expected present values.

Calculate the variance of the sum of \(Y_1\) and \(Y_2\).

(A) 3.8 (B) 3.9 (C) 4.0 (D) 4.1 (E) 4.2

6.9. Consider a special increasing 3-year temporary life annuity-due
payable annually on (x):

(i) The payment for the first year is 1000, the payment for the second
year is 3000, and the payment for the third year is 7000.

(ii) \({}_{k}p_x\) = \((0.97)^k\) for \(k\) = 0, 1, 2.

(iii) \(i\) = 0.04

(iv) \(Y\) is the present value random variable for this annuity.

Calculate the standard deviation of \(Y\).

(A) 1904 (B) 1920 (C) 1936 (D) 1952 (E) 1968

6.10. You are given a life annuity-due on (55) payable monthly. 100 is
payable each month during the first 10 years; 300 is payable each month
after the first 10 years.

Mortality follows the Illustrative Life Table, and \(i\) = 0.06.
Woolhouse's formula with two terms is used to approximate monthly
expected present values. Calculate the expected present value of this
annuity.

(A) 23,710 (B) 24,210 (C) 24,710 (D) 25,210 (E) 25,710

6.11. For a special continuous 10-year deferred life annuity on (55):

(i) Mortality follows the Illustrative Life Table, and \(i\) = 0.06.

(ii) Woolhouse's formula with two terms is used to determine annuity
expected present values.

(iii) 36,000 is payable continuously each year between ages 65 and 75.

(iv) 22,000 is payable continuously each year after age 75.

Calculate the expected present value of this annuity.

(A) 145,000 (B) 146,000 (C) 147,000 (D) 148,000 (E) 149,000

6.12. You are given:

(i) The following select-and-ultimate life table:

(ii) \(v\) = 0.97

Calculate the variance of the present value of a 3-year temporary life
annuity-due of 1000 per year on {[}35{]} payable annually.

(A) 4600 (B) 4700 (C) 4800 (D) 4900 (E) 5000

\subsection{Answers to Exercises}\label{answers-to-exercises-3}

6.1. C

6.2. A

6.3. B

6.4. E

6.5. C

6.6. D

6.7. C

6.8. B

6.9. A

6.10. D

6.11. B

6.12. C

\section{Past Exam Questions}\label{past-exam-questions-3}

\begin{itemize}
\item
  Exam MLC, Fall 2012: \#19
\item
  Exam MLC, Spring 2012: \#30
\item
  Exam MLC, Sample Questions: \#7, 284
\end{itemize}

\chapter{Premium Calculation I}\label{premium-calculation-i}

\section{Key Concepts}\label{key-concepts-2}

The policyholder often pays for a life insurance or life annuity with
multiple payments to the insurer over time called \textbf{premiums}.

\subsection{Terminology}\label{terminology}

\begin{itemize}
\item
  \textbf{Fully Continuous Insurance:} a continuous insurance that is
  funded with a continuous annuity of premiums.
\item
  \textbf{Fully Discrete Insurance:} an annual insurance that is funded
  by an annual annuity-due of premiums.
\item
  \textbf{Semi-Continuous Insurance:} a continuous insurance that is
  funded by an annual annuity-due of premiums.
\end{itemize}

\textbf{Net Loss-at-Issue}:

\begin{itemize}
\item
  The first step to determining the premiums that the policyholder
  should pay to fund the benefits of a particular policy is to determine
  the appropriate \textbf{net loss-at-issue random variable}, ignoring
  policy expenses:

  \(L_0\) = \({}_{0}L\) = \(L\) =

  Present value of future benefits at issue - Present value of future
  premiums at issue

  = \(PVFB@0\) - \(PVFP@0\)
\item
  The net loss-at-issue may be written with a superscript, as \(L^n_0\).
  ``Net'' will often be omitted if there is no expense information
  provided.
\item
  Loss is random because \(PVFB@0\) and \(PVFP@0\) each depend on the
  future lifetime of the policyholder.
\item
  There will be a \textbf{loss} on a policy if the amount the insurer
  pays out in benefits is higher than the amount the insurer collects in
  premiums; \(L_0\) \(>\) 0 if \(PVFB@0\) \(>\) \(PVFP@0\). There will
  be a \textbf{profit} on a policy if the amount the insurer pays out in
  benefits is smaller than the amount the insurer collects in premiums;
  \(L_0\) \(<\) 0 if \(PVFB@0\) \(<\) \(PVFP@0\).
\end{itemize}

\subsection{Premium Principles}\label{premium-principles}

\begin{itemize}
\item
  A \textbf{premium principle} is a rule that manipulates the
  loss-at-issue random variable in some way to generate premiums.
\item
  Premiums calculated using net loss-at-issue random variable are called
  \textbf{net premiums}.
\item
  A common premium principle is the \textbf{Equivalence Principle:}

  \begin{itemize}
  \item
    Premiums are determined such that: \(E(L_0)\) = 0.
  \item
    Under this principle, the insurer charges premiums so that, on
    average, there will be neither a loss or a profit on the policy.
    Clearly, this is not the case in practice\ldots{}
  \item
    Using the formula for \(L_0\), the equivalence principle can also be
    stated as: \(E(PVFB@0)\) = \(E(PVFP@0)\). This should be your
    starting point for complicated problems involving the equivalence
    principle.
  \item
    On Exam LC, and Exam MLC prior to 2014, net premiums determined via
    the equivalence principle are called \textbf{benefit premiums}.
    Starting in 2014, the term ``net premium'' will be equivalent to
    ``benefit premium'' on Exam MLC unless otherwise indicated; that is,
    a net premium is a premium calculated using the equivalence
    principle without expenses. In this supplement, both ``benefit
    premium'' and ``net premium'' will be used interchangeably.
  \end{itemize}
\end{itemize}

\section{Equivalence Principle}\label{equivalence-principle}

\textbf{Benefit Premiums}

\subsection{Fully Continuous Insurance of 1 on
(x)}\label{fully-continuous-insurance-of-1-on-x}

\begin{itemize}
\item
  For each fully continuous insurance,

  \(L^n_0\) = \(PVFB@0\) - (Benefit
  Premium)\(\frac{PVFP@0}{\text{Benefit
  Premium}}\), using the appropriate entries for a value of \(T_x\).
\item
  In the ``Benefit Premium'' column, the left hand side of the equals
  sign gives the actuarial notation for the benefit premium. Those
  taking Exam MLC do not have to know this notation, and can denote the
  benefit premium in each row as \(P\).
\item
  Furthermore, if the face amount is \(S\), both sides of the equation
  in the ``Benefit Premium'' column should be multiplied by \(S\).
\item
  For each fully continuous insurance, the benefit premium was
  determined by \(E(L^n_0)\) = \(E(PVFB@0)\) - (Benefit
  Premium)\(\frac{E(PVFP@0)}{\text{Benefit Premium}}\) = 0.
\item
  \(\bar{P}(\bar{A}_x)\) = \(\frac{\bar{A}_x}{\bar{a}_x}\) =
  \(\frac{\delta \bar{A}_x}{1 - \bar{A}_x}\) = \(\frac{1}{\bar{a}_x}\) -
  \(\delta\).
\item
  \(\bar{P}(\bar{A}_{x:\overline{n}|})\) =
  \(\frac{\bar{A}_{x:\overline{n}|}}{\bar{a}_{x:\overline{n}|}}\) =
  \(\frac{\delta \bar{A}_{x:\overline{n}|}}{1 - \bar{A}_{x:\overline{n}|}}\)
  = \(\frac{1}{\bar{a}_{x:\overline{n}|}}\) - \(\delta\).
\item
  With a constant force of mortality: \(\bar{P}(\bar{A}_x)\) =
  \(\bar{P}(\bar{A}_{x:\overline{n}|}^{1})\) = \(\mu\).
\item
  For a couple of the insurances in the table, there are analytic
  formulas for the variance of the net loss-at-issue.

  \begin{itemize}
  \item
    For a fully continuous whole life insurance of 1 on (x):

    \(Var(L^n_0)\) =
    \((1 + \frac{\bar{P}(\bar{A}_x)}{\delta})^2\)\((^2\bar{A}_x - [\bar{A}_x]^2)\).

    \begin{itemize}
    \tightlist
    \item
      With a constant force of mortality: \(Var(L^n_0)\) =
      \(\frac{\mu}{\mu + 2\delta}\).
    \end{itemize}
  \item
    For a fully continuous \(n\)-year endowment insurance of 1 on (x):

    \(Var(L^n_0)\) =
    \((1 + \frac{\bar{P}(\bar{A}_{x:\overline{n}|})}{\delta})^2\)\((^2\bar{A}_{x:\overline{n}|} - [\bar{A}_{x:\overline{n}|}]^2)\).
  \item
    If the benefit is \(S\), multiply each of the above \(Var(L^n_0)\)
    formulas by \(S^2\).
  \item
    These formulas for \(Var(L^n_0)\) are true for any type of premium,
    not just a benefit premium, \textbf{except} for the constant force
    of mortality formula.
  \item
    For any other type of fully continuous insurance, use:
    \(Var(L^n_0)\) = \(E[(L^n_0)^2]\) - \((E[L^n_0])^2\). If the
    equivalence principle is used to determine premiums, then:
    \(Var(L^n_0)\) = \(E[(L^n_0)^2]\).
  \end{itemize}
\item
  The equivalence principle can also determine benefit premiums for
  continuous annuities. For example:

  \(\bar{P}({}_{n|}\bar{a}_x)\) =
  \(\frac{{}_{n|}\bar{a}_x}{\bar{a}_{x:\overline{n}|}}\).
\end{itemize}

\subsection{Fully Discrete Insurance of 1 on
(x)}\label{fully-discrete-insurance-of-1-on-x}

\begin{itemize}
\item
  For each fully discrete insurance,

  \(L^n_0\) = \(PVFB@0\) - (Benefit
  Premium)\(\frac{PVFP@0}{\text{Benefit
  Premium}}\), using the appropriate entries for a value of \(K_x\).
\item
  Recall, \(K_x\) can only take on non-negative integer values. So,
  \(K_x\) \(<\) \(n\) \(\implies\) \(K_x\) = 0, 1, \ldots{}, \(n\) - 1.
\item
  In the ``Benefit Premium'' column, the left hand side of each equation
  gives the actuarial notation for the benefit premium. Those taking
  Exam MLC only have to know the notation for the whole life, \(n\)-year
  term, \(n\)-year pure endowment, and \(n\)-year endowment rows; and
  can denote other benefit premiums as \(P\).
\item
  Furthermore, if the face amount is \(S\), both sides of the equation
  in the ``Benefit Premium'' column should be multiplied by \(S\).
\item
  For each fully discrete insurance, the benefit premium was determined
  by \(E(L^n_0)\) = \(E(PVFB@0)\) - (Benefit
  Premium)\(\frac{E(PVFP@0)}{\text{Benefit Premium}}\) = 0.
\item
  \(P_x\) (the benefit premium for a fully discrete whole life insurance
  of 1 on (x)) should not be confused with \(p_x\) (the probability that
  (x) survives to age x + 1).
\item
  \(P_x\) = \(\frac{A_x}{\ddot{a}_x}\) = \(\frac{d A_x}{1 - A_x}\) =
  \(\frac{1}{\ddot{a}_x}\) - \(d\).
\item
  \(P_{x:\overline{n}|}\) =
  \(\frac{A_{x:\overline{n}|}}{\ddot{a}_{x:\overline{n}|}}\) =
  \(\frac{d A_{x:\overline{n}|}}{1 - A_{x:\overline{n}|}}\) =
  \(\frac{1}{\ddot{a}_{x:\overline{n}|}}\) - \(d\).
\item
  With a constant force of mortality: \(P_x\) =
  \({P}_{x:\overline{n}|}^{1}\) = \(v q\).
\item
  Argue that the following \textbf{3-Premium} equations are valid:

  \begin{itemize}
  \item
    \({}_{n}P_x\) - \({P}_{x:\overline{n}|}^{1}\) = \(A_{x + n}\)
    \({P}_{x:\overline{n}|}^{~~~~1}\)
  \item
    \(P_{x:\overline{n}|}\) - \({}_{n}P_x\) = {[}1 - \(A_{x + n}\){]}
    \({P}_{x:\overline{n}|}^{~~~~1}\)
  \item
    \({P}_{x:\overline{n}|}^{1}\) + \({P}_{x:\overline{n}|}^{~~~~1}\) =
    \(P_{x:\overline{n}|}\)
  \end{itemize}
\end{itemize}

\begin{itemize}
\item
  For a couple of the insurances in the table, there are analytic
  formulas for the variance of the net loss-at-issue.

  \begin{itemize}
  \item
    For a fully discrete whole life insurance of 1 on (x):

    \(Var(L^n_0)\) = \((1 + \frac{P_x}{d})^2\)\((^2A_x - [A_x]^2)\).

    \begin{itemize}
    \tightlist
    \item
      With a constant force of mortality: \(Var(L^n_0)\) =
      \(\frac{pq}{q + i^2 + 2i}\).
    \end{itemize}
  \item
    For a fully discrete \(n\)-year endowment insurance of 1 on (x):

    \(Var(L^n_0)\) =
    \((1 + \frac{P_{x:\overline{n}|}}{d})^2\)\((^2 A_{x:\overline{n}|} - [A_{x:\overline{n}|}]^2)\).
  \item
    If the benefit is \(S\), multiply each of the above \(Var(L^n_0)\)
    formulas by \(S^2\).
  \item
    These formulas for \(Var(L^n_0)\) are true for any type of premium,
    not just a benefit premium, \textbf{except} for the constant force
    of mortality formula.
  \item
    For any other type of fully discrete insurance, use: \(Var(L^n_0)\)
    = \(E[(L^n_0)^2]\) - \((E[L^n_0])^2\). If the equivalence principle
    is used to determine premiums, then: \(Var(L^n_0)\) =
    \(E[(L^n_0)^2]\).
  \end{itemize}
\item
  The equivalence principle can also determine benefit premiums for
  discrete annuities. For example:

  \(P({}_{n|}\ddot{a}_x)\) =
  \(\frac{{}_{n|}\ddot{a}_x}{\ddot{a}_{x:\overline{n}|}}\).
\end{itemize}

\subsection{Semi-Continuous Insurance of 1 on
(x)}\label{semi-continuous-insurance-of-1-on-x}

\begin{itemize}
\item
  You can obtain this table by taking the table for \textbf{Fully
  Continuous Insurance of 1 on (x)} and replacing the continuous premium
  annuity with an annual annuity-due.
\item
  For example, a semi-continuous \(n\)-year term insurance of 1 on (x)
  is:
\item
  Those taking Exam MLC do NOT have to know the actuarial notation for
  semi-continuous benefit premiums; \(P\) is sufficient.
\item
  Exam MLC Only: With a uniform distribution of deaths (UDD) in each
  year of age:

  \begin{itemize}
  \item
    \(P(\bar{A}_x)\) = \(\frac{i}{\delta}\)\(P_x\)
  \item
    \(P(\bar{A}_{x:\overline{n}|}^{1})\) =
    \(\frac{i}{\delta}\)\({P}_{x:\overline{n}|}^{1}\)
  \item
    \(P(\bar{A}_{x:\overline{n}|})\) =
    \(\frac{i}{\delta}\)\({P}_{x:\overline{n}|}^{1}\) +
    \({P}_{x:\overline{n}|}^{~~~~1}\)
  \end{itemize}
\end{itemize}

\section{Exercises}\label{exercises-4}

7.1. On January 1, 2010, Pat purchases a 5-year deferred whole life
insurance of 100,000 payable at the end of the year of death. Premiums
of 4000 are payable at the beginning of each year for the first 5 years,
and \(i\) = 0.05.

Calculate the loss-at-issue if Pat dies on September 30, 2016.

(A) 48,370 (B) 52,884 (C) 53,756 (D) 57,209 (E) 62,187

7.2. Stefano, age 60, purchases a whole life insurance of 1,000,000:

(i) The death benefit is payable at the moment of death.

(ii) Premiums of 50,000 are payable at the beginning of each year for as
long as Stefano is alive.

(iii) \(i\) = 0.05

(iv) \(L\) is the loss-at-issue random variable.

Calculate the value of \(L\) if Stefano dies at age 61.5.

(A) 764,059 (B) 809,410 (C) 819,138 (D) 831,810 (E) 879,429

7.3. Consider a fully continuous whole life insurance of 1000 on (x).

Assume \(\delta\) = 0.08 and \(\mu_x(t)\) = 0.04 for \(t\) \(\ge\) 0.

Calculate the level annual benefit premium.

(A) 30 (B) 35 (C) 40 (D) 45 (E) 50

7.4. Paul, age 31, purchases a fully discrete 20-year endowment
insurance of 1000. Assume mortality follows the Illustrative Life Table,
and \(i\) = 0.06.

Calculate: 1000\(P_{31 :\overline{20}|}\).

(A) 21 (B) 23 (C) 25 (D) 27 (E) 29

7.5. Consider the following special fully discrete whole life insurance
on (50):

(i) The death benefit is 80,000 before age 65; the death benefit is
150,000 thereafter.

(ii) The level annual net premium is \(2P\) before age 65; the level
annual net premium is \(P\) thereafter.

(iii) Mortality follows the Illustrative Life Table, and \(i\) = 0.06.

Calculate the annual net premium payable before age 65.

(A) 1300 (B) 1800 (C) 2300 (D) 2700 (E) 3200

7.6. Consider a fully discrete 5-payment 10-year endowment insurance of
1000 on (70):

(i) Mortality follows the Illustrative Life Table.

(ii) \(i\) = 0.06

Calculate the level annual net premium.

(A) 140 (B) 145 (C) 150 (D) 155 (E) 160

7.7. For a fully continuous whole life insurance of 5000 on (x):

(i) The force of mortality is a constant.

(ii) \(\delta\) = 0.05

(iii) \(L\) is the loss-at-issue random variable based on level annual
benefit premiums.

(iv) The standard deviation of \(L\) is 2236.07

Calculate the level annual benefit premium.

(A) 100 (B) 125 (C) 150 (D) 175 (E) 200

7.8. A fully continuous whole life insurance of 10,000 on (x) is issued
with premiums determined by the equivalence principle.

You are also given:

(i) \(\mu_x(t)\) = 0.02 for \(t\) \(\ge\) 0

(ii) \(\delta\) = 0.05

Calculate the probability that the loss-at-issue is positive.

(A) 0.30 (B) 0.33 (C) 0.36 (D) 0.39 (E) 0.42

7.9. An insurer has just issued each of 100 independent lives aged 35 a
fully discrete 20-year endowment insurance of 1000 with level annual
benefit premiums. Each life has mortality that follows the Illustrative
Life Table. The effective annual interest rate is 0.06.

Using the normal approximation, determine the fund amount at issue,
\(h\), that is necessary so that the insurer is 99\% sure that the sum
of the 100 loss-at-issue random variables associated with the endowment
insurances will not exceed \(h\).

(A) 2100 (B) 2200 (C) 2300 (D) 2400 (E) 2500

7.10. For a special fully discrete whole life insurance on (35):

(i) The death benefit is equal to 2000 plus the return of all benefit
premiums paid in the past without interest.

(ii) \(\ddot{a}_{35}\) = 19.93

(iii) \((IA)_{35}\) = 5.58

(iv) \(i\) = 0.045

Calculate the level annual benefit premium for this insurance.

(A) 20 (B) 22 (C) 24 (D) 26 (E) 28

7.11. You are given:

(i) The level annual benefit premium for a fully discrete 20-year term
insurance of 5000 on (x) is 75.

(ii) The level annual benefit premium for a fully discrete 20-year
endowment insurance of 5000 on (x) is 200.

(iii) The level annual benefit premium for a fully discrete 20-payment
whole life insurance of 5000 on (x) is 150.

Calculate the actuarial present value of a fully discrete whole life
insurance of 5000 on (x + 20).

(A) 2000 (B) 2500 (C) 3000 (D) 3500 (E) 4000

7.12. For a life insurance on (x):

(i) 1000 is payable at the end of the year of death if death occurs in
the first ten years; 2000 is payable at the end of the year of death if
death occurs in the next ten years; otherwise, the death benefit is 0.

(ii) Level annual benefit premiums are payable at the beginning of each
year for the first 20 years.

(iii) \(d\) = 0.10, and \(q_x\) = 0.03 for all integer ages x.

Calculate the level annual benefit premium.

(A) 31 (B) 33 (C) 35 (D) 37 (E) 39

7.13. For a special fully discrete 3-year term insurance on (x):

(i) The death benefit: \(b_{k + 1}\) = 500(\(k\) + 1) for \(k\) = 0, 1,
2

(ii) \(q_{x + k}\) = 0.02(\(k\) + 1) for \(k\) = 0, 1, 2

(iii) \(i\) = 0.03

Use the equivalence principle to calculate the level annual premium for
this insurance.

(A) 41 (B) 42 (C) 43 (D) 44 (E) 45

7.14. You are given:

(i) \(l_x\) = 100\((110 - x)^2\) for 0 \(\le\) \(x\) \(\le\) 110

(ii) \(i\) = 0

Calculate the level annual benefit premium for a fully discrete
5-payment 15-year term insurance of 1 on (30):
\({}_{5} {P}_{30:\overline{15}|}^{1}\).

(A) 0.06 (B) 0.07 (C) 0.08 (D) 0.09 (E) 0.10

7.15. A fully discrete 5-year endowment insurance of 1000 was just
issued to Math Mage, aged 30. In determining the level annual benefit
premium, it was assumed that \(i\) = 0.06 and that Math Mage had
mortality that follows the Illustrative Life Table.

Shortly after issuing the 5-year endowment insurance, it was discovered
that Math Mage had been cursed by Hattendorf. In calculating the level
annual benefit premium, it should have been assumed that \(i\) = 0.06
and that Math Mage had mortality such that the actual force of mortality
was \(\mu_{30}(t)\) + 0.10 for 0 \(<\) \(t\) \(<\) 5, where
\(\mu_{30}(t)\) is the force of mortality associated with the
Illustrative Life Table.

Calculate the difference between the benefit premium that Math Mage
should be paying calculated using the correct mortality (based on
\(\mu_{30}(t)\) + 0.10) and the benefit premium actually payable by Math
Mage calculated using the incorrect mortality (Illustrative Life Table).

(A) 40 (B) 50 (C) 60 (D) 70 (E) 80

7.16. Consider a 5-year deferred whole life annuity-due on (60) with an
annual payment of 10,000. You are given:

(i) Benefit premiums are payable at the beginning of each year during
the first five years. The benefit premium payable in each of years one
and two is half of the benefit premium payable in each of years three,
four, and five.

(ii) \(d\) = 0.04306

(iii) \(q_{60}\) = 0.006155 and \(q_{61}\) = 0.006765

(iv) \({}_{5}E_{60}\) = 0.77282

(v) \(\ddot{a}_{65}\) = 13.4662 and \(\ddot{a}_{62:\overline{3}|}\) =
2.8513

Calculate the benefit premium payable in each of years one and two.

(A) 14,300 (B) 14,600 (C) 14,900 (D) 15,200 (E) 15,500

7.17. Consider a fully continuous whole life insurance of 100,000 on
(30):

(i) If the level annual premium is \(\pi_1\), the standard deviation of
the loss-at-issue random variable is 55,621.49.

(ii) If the level annual premium is \(\pi_2\), the standard deviation of
the loss-at-issue random variable is 49,441.32.

(iii) \(\pi_1\) is 1.5 times \(\pi_2\).

(iv) \(\delta\) = 0.04

(v) \(Z\) is the present value random variable for the continuous whole
life insurance of 100,000 on (30).

Calculate the standard deviation of \(Z\).

(A) 37,100 (B) 38,200 (C) 39,300 (D) 40,400 (E) 41,500

7.18. Bruce and Lucius, both aged x, have each just purchased a fully
discrete 3-year term insurance of 1000:

(i) Bruce pays a benefit premium of 175.72 each year. If Bruce dies in
the second year after policy issue, the loss-at-issue is 559.27.

(ii) Lucius pays non-level annual benefit premiums. The first benefit
premium is 100, the second benefit premium is 175, and the third benefit
premium is \(P\).

(iii) Each life has mortality such that: \({}_{k|}q_x\) =
\((0.3)^{k + 1}\) for \(k\) = 0, 1, 2.

(iv) The effective annual interest rate is \(i\).

Calculate: \(P\).

(A) 285 (B) 315 (C) 345 (D) 375 (E) 405

7.19. Consider a special 20-year deferred whole life annuity-due of 5000
per year on (45) payable annually:

(i) Level annual benefit premiums are payable at the beginning of the
year during the first 20 years after policy issue.

(ii) There is a death benefit during the premium-paying period, payable
at the end of the year of death, that is equal to the return of all
benefit premiums previously paid with interest at 6\%.

(iii) \(i\) = 0.06

(iv) Mortality follows the Illustrative Life Table.

Calculate the benefit premium.

(A) 1180 (B) 1210 (C) 1240 (D) 1270 (E) 1300

7.20. You are given:

(i) A fully discrete 2-year deferred, 3-year term insurance of 1000 is
issued to a life aged x.

(ii) Level annual premiums are only payable during the first two years.

(iii) The level annual premium is determined such that the average
loss-at-issue is zero.

(iv) \(v\) = 0.90

(v) \({}_{k|}q_x\) = 0.05(1 + \(k\)) for \(k\) = 0, 1, 2, 3, 4;
\({}_{5|}q_x\) = 0.25

Calculate the median loss-at-issue.

(A) 193 (B) 226 (C) 258 (D) 295 (E) 331

7.21. Paul is attempting to determine the level annual benefit premium
for a fully discrete 20-year endowment insurance of 10,000 on (55). Paul
assumes the following:

(i) \(i\) = 0.06

(ii) Paul is not sure of the future lifetime distribution of (55). He
believes there is an 80\% probability that the mortality of (55) follows
the Illustrative Life Table, and that there is a 20\% probability that
the mortality of (55) is such that \({}_{k|}q_{55}\) = 0.05 for \(k\) =
0, 1, 2, \ldots{}, 19.

Determine the level annual benefit premium.

(A) 384 (B) 414 (C) 444 (D) 474 (E) 504

7.22. Suppose today is January 1, 2014, and Paul has just turned age 35.
He has mortality such that:

\({}_{t}p_{35}\) = \((0.95)^t\) for \(t\) \(\ge\) 0.

Paul has just been issued a special fully discrete whole life insurance:

(i) \(i\) = 0.07

(ii) The death benefit if Paul dies in an odd-numbered calendar year is
5000.

(iii) The death benefit if Paul dies in an even-numbered calendar year
is 10,000.

(iv) Benefit premiums are payable annually, where the benefit premium
for an odd-numbered calendar year is double the benefit premium for an
even-numbered calendar year.

Calculate the benefit premium payable during an even-numbered calendar
year.

(A) 220 (B) 240 (C) 260 (D) 280 (E) 300

\subsection{Answers to Exercises}\label{answers-to-exercises-4}

7.1. B

7.2. D

7.3. C

7.4. D

7.5. D

7.6. C

7.7. B

7.8. D

7.9. E

7.10. A

7.11. C

7.12. B

7.13. D

7.14. B

7.15. A

7.16. B

7.17. A

7.18. B

7.19. D

7.20. C

7.21. B

7.22. B

\section{Past Exam Questions}\label{past-exam-questions-4}

\begin{itemize}
\item
  Exam MLC, Fall 2015: \#10
\item
  Exam MLC, Spring 2015: \#8, 11
\item
  Exam MLC, Fall 2014: \#7
\item
  Exam MLC, Spring 2014: \#8
\item
  Exam MLC, Fall 2013: \#14, 15, 16
\item
  Exam MLC, Spring 2013: \#1, 3, 18
\item
  Exam 3L, Spring 2013: \#13, 14
\item
  Exam MLC, Fall 2012: \#20, 22
\item
  Exam 3L, Spring 2012: \#12
\item
  Exam MLC, Spring 2012: \#3, 6
\item
  Exam MLC, Sample Questions: \#6, 14, 29, 40, 47, 51, 76, 84, 92, 96,
  97, 99, 111, 119, 127, 142, 154, 157, 172, 174, 184, 204, 221, 228,
  309
\item
  Exam 3L, Spring 2010: \#16
\item
  Exam 3L, Fall 2008: \#22
\item
  Exam 3L, Spring 2008: \#23
\item
  Exam MLC, Spring 2007: \#4
\end{itemize}

\bibliography{packages,LDAReference}


\end{document}
